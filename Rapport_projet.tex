% Options for packages loaded elsewhere
\PassOptionsToPackage{unicode}{hyperref}
\PassOptionsToPackage{hyphens}{url}
%
\documentclass[
]{article}
\usepackage{amsmath,amssymb}
\usepackage{iftex}
\ifPDFTeX
  \usepackage[T1]{fontenc}
  \usepackage[utf8]{inputenc}
  \usepackage{textcomp} % provide euro and other symbols
\else % if luatex or xetex
  \usepackage{unicode-math} % this also loads fontspec
  \defaultfontfeatures{Scale=MatchLowercase}
  \defaultfontfeatures[\rmfamily]{Ligatures=TeX,Scale=1}
\fi
\usepackage{lmodern}
\ifPDFTeX\else
  % xetex/luatex font selection
\fi
% Use upquote if available, for straight quotes in verbatim environments
\IfFileExists{upquote.sty}{\usepackage{upquote}}{}
\IfFileExists{microtype.sty}{% use microtype if available
  \usepackage[]{microtype}
  \UseMicrotypeSet[protrusion]{basicmath} % disable protrusion for tt fonts
}{}
\makeatletter
\@ifundefined{KOMAClassName}{% if non-KOMA class
  \IfFileExists{parskip.sty}{%
    \usepackage{parskip}
  }{% else
    \setlength{\parindent}{0pt}
    \setlength{\parskip}{6pt plus 2pt minus 1pt}}
}{% if KOMA class
  \KOMAoptions{parskip=half}}
\makeatother
\usepackage{xcolor}
\usepackage[margin=1in]{geometry}
\usepackage{color}
\usepackage{fancyvrb}
\newcommand{\VerbBar}{|}
\newcommand{\VERB}{\Verb[commandchars=\\\{\}]}
\DefineVerbatimEnvironment{Highlighting}{Verbatim}{commandchars=\\\{\}}
% Add ',fontsize=\small' for more characters per line
\usepackage{framed}
\definecolor{shadecolor}{RGB}{248,248,248}
\newenvironment{Shaded}{\begin{snugshade}}{\end{snugshade}}
\newcommand{\AlertTok}[1]{\textcolor[rgb]{0.94,0.16,0.16}{#1}}
\newcommand{\AnnotationTok}[1]{\textcolor[rgb]{0.56,0.35,0.01}{\textbf{\textit{#1}}}}
\newcommand{\AttributeTok}[1]{\textcolor[rgb]{0.13,0.29,0.53}{#1}}
\newcommand{\BaseNTok}[1]{\textcolor[rgb]{0.00,0.00,0.81}{#1}}
\newcommand{\BuiltInTok}[1]{#1}
\newcommand{\CharTok}[1]{\textcolor[rgb]{0.31,0.60,0.02}{#1}}
\newcommand{\CommentTok}[1]{\textcolor[rgb]{0.56,0.35,0.01}{\textit{#1}}}
\newcommand{\CommentVarTok}[1]{\textcolor[rgb]{0.56,0.35,0.01}{\textbf{\textit{#1}}}}
\newcommand{\ConstantTok}[1]{\textcolor[rgb]{0.56,0.35,0.01}{#1}}
\newcommand{\ControlFlowTok}[1]{\textcolor[rgb]{0.13,0.29,0.53}{\textbf{#1}}}
\newcommand{\DataTypeTok}[1]{\textcolor[rgb]{0.13,0.29,0.53}{#1}}
\newcommand{\DecValTok}[1]{\textcolor[rgb]{0.00,0.00,0.81}{#1}}
\newcommand{\DocumentationTok}[1]{\textcolor[rgb]{0.56,0.35,0.01}{\textbf{\textit{#1}}}}
\newcommand{\ErrorTok}[1]{\textcolor[rgb]{0.64,0.00,0.00}{\textbf{#1}}}
\newcommand{\ExtensionTok}[1]{#1}
\newcommand{\FloatTok}[1]{\textcolor[rgb]{0.00,0.00,0.81}{#1}}
\newcommand{\FunctionTok}[1]{\textcolor[rgb]{0.13,0.29,0.53}{\textbf{#1}}}
\newcommand{\ImportTok}[1]{#1}
\newcommand{\InformationTok}[1]{\textcolor[rgb]{0.56,0.35,0.01}{\textbf{\textit{#1}}}}
\newcommand{\KeywordTok}[1]{\textcolor[rgb]{0.13,0.29,0.53}{\textbf{#1}}}
\newcommand{\NormalTok}[1]{#1}
\newcommand{\OperatorTok}[1]{\textcolor[rgb]{0.81,0.36,0.00}{\textbf{#1}}}
\newcommand{\OtherTok}[1]{\textcolor[rgb]{0.56,0.35,0.01}{#1}}
\newcommand{\PreprocessorTok}[1]{\textcolor[rgb]{0.56,0.35,0.01}{\textit{#1}}}
\newcommand{\RegionMarkerTok}[1]{#1}
\newcommand{\SpecialCharTok}[1]{\textcolor[rgb]{0.81,0.36,0.00}{\textbf{#1}}}
\newcommand{\SpecialStringTok}[1]{\textcolor[rgb]{0.31,0.60,0.02}{#1}}
\newcommand{\StringTok}[1]{\textcolor[rgb]{0.31,0.60,0.02}{#1}}
\newcommand{\VariableTok}[1]{\textcolor[rgb]{0.00,0.00,0.00}{#1}}
\newcommand{\VerbatimStringTok}[1]{\textcolor[rgb]{0.31,0.60,0.02}{#1}}
\newcommand{\WarningTok}[1]{\textcolor[rgb]{0.56,0.35,0.01}{\textbf{\textit{#1}}}}
\usepackage{longtable,booktabs,array}
\usepackage{calc} % for calculating minipage widths
% Correct order of tables after \paragraph or \subparagraph
\usepackage{etoolbox}
\makeatletter
\patchcmd\longtable{\par}{\if@noskipsec\mbox{}\fi\par}{}{}
\makeatother
% Allow footnotes in longtable head/foot
\IfFileExists{footnotehyper.sty}{\usepackage{footnotehyper}}{\usepackage{footnote}}
\makesavenoteenv{longtable}
\usepackage{graphicx}
\makeatletter
\def\maxwidth{\ifdim\Gin@nat@width>\linewidth\linewidth\else\Gin@nat@width\fi}
\def\maxheight{\ifdim\Gin@nat@height>\textheight\textheight\else\Gin@nat@height\fi}
\makeatother
% Scale images if necessary, so that they will not overflow the page
% margins by default, and it is still possible to overwrite the defaults
% using explicit options in \includegraphics[width, height, ...]{}
\setkeys{Gin}{width=\maxwidth,height=\maxheight,keepaspectratio}
% Set default figure placement to htbp
\makeatletter
\def\fps@figure{htbp}
\makeatother
\setlength{\emergencystretch}{3em} % prevent overfull lines
\providecommand{\tightlist}{%
  \setlength{\itemsep}{0pt}\setlength{\parskip}{0pt}}
\setcounter{secnumdepth}{-\maxdimen} % remove section numbering
\usepackage{pdfpages} \usepackage{graphics}
\usepackage{multirow}
\usepackage{multicol}
\usepackage{colortbl}
\usepackage{hhline}
\newlength\Oldarrayrulewidth
\newlength\Oldtabcolsep
\usepackage{longtable}
\usepackage{array}
\usepackage{hyperref}
\usepackage{float}
\usepackage{wrapfig}
\ifLuaTeX
  \usepackage{selnolig}  % disable illegal ligatures
\fi
\IfFileExists{bookmark.sty}{\usepackage{bookmark}}{\usepackage{hyperref}}
\IfFileExists{xurl.sty}{\usepackage{xurl}}{} % add URL line breaks if available
\urlstyle{same}
\hypersetup{
  hidelinks,
  pdfcreator={LaTeX via pandoc}}

\author{}
\date{\vspace{-2.5em}}

\begin{document}

\includepdf{page_de_garde.pdf}

\tableofcontents

\hypertarget{section}{%
\section{\texorpdfstring{\textcolor{blue}{PARTIE 1}}{}}\label{section}}

\hypertarget{i.-pruxe9paration-des-donnuxe9es}{%
\subsection{I. Préparation des
données}\label{i.-pruxe9paration-des-donnuxe9es}}

\hypertarget{i.1-description-de-la-base}{%
\subsubsection{I.1 Description de la
base}\label{i.1-description-de-la-base}}

\hypertarget{i.2-importation-et-mise-en-forme}{%
\subsubsection{\texorpdfstring{\textbf{I.2 Importation et mise en
forme}}{I.2 Importation et mise en forme}}\label{i.2-importation-et-mise-en-forme}}

\hypertarget{importation-des-bibliothuxe8ques-nuxe9cessaires}{%
\paragraph{Importation des bibliothèques
nécessaires}\label{importation-des-bibliothuxe8ques-nuxe9cessaires}}

\begin{Shaded}
\begin{Highlighting}[]
\CommentTok{\#Importation des librairies nécessaires}
\FunctionTok{library}\NormalTok{(}\StringTok{"readxl"}\NormalTok{)}
\FunctionTok{library}\NormalTok{(}\StringTok{"gtsummary"}\NormalTok{)}
\FunctionTok{library}\NormalTok{(}\StringTok{"flextable"}\NormalTok{)}
\FunctionTok{library}\NormalTok{(}\StringTok{"dplyr"}\NormalTok{)}
\FunctionTok{library}\NormalTok{(}\StringTok{"ggplot2"}\NormalTok{)}
\FunctionTok{library}\NormalTok{(}\StringTok{"sf"}\NormalTok{)}
\FunctionTok{library}\NormalTok{(}\StringTok{"leaflet"}\NormalTok{)}
\FunctionTok{library}\NormalTok{(}\StringTok{"DT"}\NormalTok{)}
\FunctionTok{library}\NormalTok{(}\StringTok{"ggspatial"}\NormalTok{)}
\FunctionTok{library}\NormalTok{(}\StringTok{"webshot"}\NormalTok{)}
\FunctionTok{library}\NormalTok{(}\StringTok{"gt"}\NormalTok{)}
\end{Highlighting}
\end{Shaded}

\hypertarget{importation-de-la-base-de-donnuxe9es-dans-un-objet-de-type-data.frame-nommuxe9-projet}{%
\paragraph{Importation de la base de données dans un objet de type
data.frame nommé
projet}\label{importation-de-la-base-de-donnuxe9es-dans-un-objet-de-type-data.frame-nommuxe9-projet}}

\begin{Shaded}
\begin{Highlighting}[]
\CommentTok{\#Importation de la base de données dans un objet de type data.frame nommé projet}
\NormalTok{projet}\OtherTok{\textless{}{-}}\FunctionTok{read\_xlsx}\NormalTok{(}\StringTok{"Base\_Partie 1.xlsx"}\NormalTok{)}
\end{Highlighting}
\end{Shaded}

\hypertarget{tableau-ruxe9sumant-les-valeurs-manquantes-par-variable}{%
\paragraph{Tableau résumant les valeurs manquantes par
variable}\label{tableau-ruxe9sumant-les-valeurs-manquantes-par-variable}}

\begin{Shaded}
\begin{Highlighting}[]
\CommentTok{\#Tableau résumant les valeurs manquantes par variable}
\NormalTok{flextable}\SpecialCharTok{::}\FunctionTok{flextable}\NormalTok{(}\FunctionTok{data.frame}\NormalTok{(}\AttributeTok{variables=}\FunctionTok{colnames}\NormalTok{(projet),}
                                \StringTok{\textquotesingle{}valeurs manquantes\textquotesingle{}}\OtherTok{=}\FunctionTok{apply}\NormalTok{(projet, }\AttributeTok{MARGIN =} \DecValTok{2}\NormalTok{, }
                                                       \ControlFlowTok{function}\NormalTok{(x) }\FunctionTok{sum}\NormalTok{(}\FunctionTok{is.na}\NormalTok{(x)))))}
\end{Highlighting}
\end{Shaded}

\global\setlength{\Oldarrayrulewidth}{\arrayrulewidth}

\global\setlength{\Oldtabcolsep}{\tabcolsep}

\setlength{\tabcolsep}{0pt}

\renewcommand*{\arraystretch}{1.5}



\providecommand{\ascline}[3]{\noalign{\global\arrayrulewidth #1}\arrayrulecolor[HTML]{#2}\cline{#3}}

\begin{longtable}[c]{|p{0.75in}|p{0.75in}}



\ascline{1.5pt}{666666}{1-2}

\multicolumn{1}{>{\raggedright}m{\dimexpr 0.75in+0\tabcolsep}}{\textcolor[HTML]{000000}{\fontsize{11}{11}\selectfont{variables}}} & \multicolumn{1}{>{\raggedleft}m{\dimexpr 0.75in+0\tabcolsep}}{\textcolor[HTML]{000000}{\fontsize{11}{11}\selectfont{valeurs.manquantes}}} \\

\ascline{1.5pt}{666666}{1-2}\endfirsthead 

\ascline{1.5pt}{666666}{1-2}

\multicolumn{1}{>{\raggedright}m{\dimexpr 0.75in+0\tabcolsep}}{\textcolor[HTML]{000000}{\fontsize{11}{11}\selectfont{variables}}} & \multicolumn{1}{>{\raggedleft}m{\dimexpr 0.75in+0\tabcolsep}}{\textcolor[HTML]{000000}{\fontsize{11}{11}\selectfont{valeurs.manquantes}}} \\

\ascline{1.5pt}{666666}{1-2}\endhead



\multicolumn{1}{>{\raggedright}m{\dimexpr 0.75in+0\tabcolsep}}{\textcolor[HTML]{000000}{\fontsize{11}{11}\selectfont{key}}} & \multicolumn{1}{>{\raggedleft}m{\dimexpr 0.75in+0\tabcolsep}}{\textcolor[HTML]{000000}{\fontsize{11}{11}\selectfont{0}}} \\





\multicolumn{1}{>{\raggedright}m{\dimexpr 0.75in+0\tabcolsep}}{\textcolor[HTML]{000000}{\fontsize{11}{11}\selectfont{q1}}} & \multicolumn{1}{>{\raggedleft}m{\dimexpr 0.75in+0\tabcolsep}}{\textcolor[HTML]{000000}{\fontsize{11}{11}\selectfont{0}}} \\





\multicolumn{1}{>{\raggedright}m{\dimexpr 0.75in+0\tabcolsep}}{\textcolor[HTML]{000000}{\fontsize{11}{11}\selectfont{q2}}} & \multicolumn{1}{>{\raggedleft}m{\dimexpr 0.75in+0\tabcolsep}}{\textcolor[HTML]{000000}{\fontsize{11}{11}\selectfont{0}}} \\





\multicolumn{1}{>{\raggedright}m{\dimexpr 0.75in+0\tabcolsep}}{\textcolor[HTML]{000000}{\fontsize{11}{11}\selectfont{q23}}} & \multicolumn{1}{>{\raggedleft}m{\dimexpr 0.75in+0\tabcolsep}}{\textcolor[HTML]{000000}{\fontsize{11}{11}\selectfont{0}}} \\





\multicolumn{1}{>{\raggedright}m{\dimexpr 0.75in+0\tabcolsep}}{\textcolor[HTML]{000000}{\fontsize{11}{11}\selectfont{q24}}} & \multicolumn{1}{>{\raggedleft}m{\dimexpr 0.75in+0\tabcolsep}}{\textcolor[HTML]{000000}{\fontsize{11}{11}\selectfont{0}}} \\





\multicolumn{1}{>{\raggedright}m{\dimexpr 0.75in+0\tabcolsep}}{\textcolor[HTML]{000000}{\fontsize{11}{11}\selectfont{q24a\_1}}} & \multicolumn{1}{>{\raggedleft}m{\dimexpr 0.75in+0\tabcolsep}}{\textcolor[HTML]{000000}{\fontsize{11}{11}\selectfont{0}}} \\





\multicolumn{1}{>{\raggedright}m{\dimexpr 0.75in+0\tabcolsep}}{\textcolor[HTML]{000000}{\fontsize{11}{11}\selectfont{q24a\_2}}} & \multicolumn{1}{>{\raggedleft}m{\dimexpr 0.75in+0\tabcolsep}}{\textcolor[HTML]{000000}{\fontsize{11}{11}\selectfont{0}}} \\





\multicolumn{1}{>{\raggedright}m{\dimexpr 0.75in+0\tabcolsep}}{\textcolor[HTML]{000000}{\fontsize{11}{11}\selectfont{q24a\_3}}} & \multicolumn{1}{>{\raggedleft}m{\dimexpr 0.75in+0\tabcolsep}}{\textcolor[HTML]{000000}{\fontsize{11}{11}\selectfont{0}}} \\





\multicolumn{1}{>{\raggedright}m{\dimexpr 0.75in+0\tabcolsep}}{\textcolor[HTML]{000000}{\fontsize{11}{11}\selectfont{q24a\_4}}} & \multicolumn{1}{>{\raggedleft}m{\dimexpr 0.75in+0\tabcolsep}}{\textcolor[HTML]{000000}{\fontsize{11}{11}\selectfont{0}}} \\





\multicolumn{1}{>{\raggedright}m{\dimexpr 0.75in+0\tabcolsep}}{\textcolor[HTML]{000000}{\fontsize{11}{11}\selectfont{q24a\_5}}} & \multicolumn{1}{>{\raggedleft}m{\dimexpr 0.75in+0\tabcolsep}}{\textcolor[HTML]{000000}{\fontsize{11}{11}\selectfont{0}}} \\





\multicolumn{1}{>{\raggedright}m{\dimexpr 0.75in+0\tabcolsep}}{\textcolor[HTML]{000000}{\fontsize{11}{11}\selectfont{q24a\_6}}} & \multicolumn{1}{>{\raggedleft}m{\dimexpr 0.75in+0\tabcolsep}}{\textcolor[HTML]{000000}{\fontsize{11}{11}\selectfont{0}}} \\





\multicolumn{1}{>{\raggedright}m{\dimexpr 0.75in+0\tabcolsep}}{\textcolor[HTML]{000000}{\fontsize{11}{11}\selectfont{q24a\_7}}} & \multicolumn{1}{>{\raggedleft}m{\dimexpr 0.75in+0\tabcolsep}}{\textcolor[HTML]{000000}{\fontsize{11}{11}\selectfont{0}}} \\





\multicolumn{1}{>{\raggedright}m{\dimexpr 0.75in+0\tabcolsep}}{\textcolor[HTML]{000000}{\fontsize{11}{11}\selectfont{q24a\_9}}} & \multicolumn{1}{>{\raggedleft}m{\dimexpr 0.75in+0\tabcolsep}}{\textcolor[HTML]{000000}{\fontsize{11}{11}\selectfont{0}}} \\





\multicolumn{1}{>{\raggedright}m{\dimexpr 0.75in+0\tabcolsep}}{\textcolor[HTML]{000000}{\fontsize{11}{11}\selectfont{q24a\_10}}} & \multicolumn{1}{>{\raggedleft}m{\dimexpr 0.75in+0\tabcolsep}}{\textcolor[HTML]{000000}{\fontsize{11}{11}\selectfont{0}}} \\





\multicolumn{1}{>{\raggedright}m{\dimexpr 0.75in+0\tabcolsep}}{\textcolor[HTML]{000000}{\fontsize{11}{11}\selectfont{q25}}} & \multicolumn{1}{>{\raggedleft}m{\dimexpr 0.75in+0\tabcolsep}}{\textcolor[HTML]{000000}{\fontsize{11}{11}\selectfont{0}}} \\





\multicolumn{1}{>{\raggedright}m{\dimexpr 0.75in+0\tabcolsep}}{\textcolor[HTML]{000000}{\fontsize{11}{11}\selectfont{q26}}} & \multicolumn{1}{>{\raggedleft}m{\dimexpr 0.75in+0\tabcolsep}}{\textcolor[HTML]{000000}{\fontsize{11}{11}\selectfont{0}}} \\





\multicolumn{1}{>{\raggedright}m{\dimexpr 0.75in+0\tabcolsep}}{\textcolor[HTML]{000000}{\fontsize{11}{11}\selectfont{q12}}} & \multicolumn{1}{>{\raggedleft}m{\dimexpr 0.75in+0\tabcolsep}}{\textcolor[HTML]{000000}{\fontsize{11}{11}\selectfont{0}}} \\





\multicolumn{1}{>{\raggedright}m{\dimexpr 0.75in+0\tabcolsep}}{\textcolor[HTML]{000000}{\fontsize{11}{11}\selectfont{q14b}}} & \multicolumn{1}{>{\raggedleft}m{\dimexpr 0.75in+0\tabcolsep}}{\textcolor[HTML]{000000}{\fontsize{11}{11}\selectfont{1}}} \\





\multicolumn{1}{>{\raggedright}m{\dimexpr 0.75in+0\tabcolsep}}{\textcolor[HTML]{000000}{\fontsize{11}{11}\selectfont{q16}}} & \multicolumn{1}{>{\raggedleft}m{\dimexpr 0.75in+0\tabcolsep}}{\textcolor[HTML]{000000}{\fontsize{11}{11}\selectfont{1}}} \\





\multicolumn{1}{>{\raggedright}m{\dimexpr 0.75in+0\tabcolsep}}{\textcolor[HTML]{000000}{\fontsize{11}{11}\selectfont{q17}}} & \multicolumn{1}{>{\raggedleft}m{\dimexpr 0.75in+0\tabcolsep}}{\textcolor[HTML]{000000}{\fontsize{11}{11}\selectfont{131}}} \\





\multicolumn{1}{>{\raggedright}m{\dimexpr 0.75in+0\tabcolsep}}{\textcolor[HTML]{000000}{\fontsize{11}{11}\selectfont{q19}}} & \multicolumn{1}{>{\raggedleft}m{\dimexpr 0.75in+0\tabcolsep}}{\textcolor[HTML]{000000}{\fontsize{11}{11}\selectfont{120}}} \\





\multicolumn{1}{>{\raggedright}m{\dimexpr 0.75in+0\tabcolsep}}{\textcolor[HTML]{000000}{\fontsize{11}{11}\selectfont{q20}}} & \multicolumn{1}{>{\raggedleft}m{\dimexpr 0.75in+0\tabcolsep}}{\textcolor[HTML]{000000}{\fontsize{11}{11}\selectfont{0}}} \\





\multicolumn{1}{>{\raggedright}m{\dimexpr 0.75in+0\tabcolsep}}{\textcolor[HTML]{000000}{\fontsize{11}{11}\selectfont{filiere\_1}}} & \multicolumn{1}{>{\raggedleft}m{\dimexpr 0.75in+0\tabcolsep}}{\textcolor[HTML]{000000}{\fontsize{11}{11}\selectfont{0}}} \\





\multicolumn{1}{>{\raggedright}m{\dimexpr 0.75in+0\tabcolsep}}{\textcolor[HTML]{000000}{\fontsize{11}{11}\selectfont{filiere\_2}}} & \multicolumn{1}{>{\raggedleft}m{\dimexpr 0.75in+0\tabcolsep}}{\textcolor[HTML]{000000}{\fontsize{11}{11}\selectfont{0}}} \\





\multicolumn{1}{>{\raggedright}m{\dimexpr 0.75in+0\tabcolsep}}{\textcolor[HTML]{000000}{\fontsize{11}{11}\selectfont{filiere\_3}}} & \multicolumn{1}{>{\raggedleft}m{\dimexpr 0.75in+0\tabcolsep}}{\textcolor[HTML]{000000}{\fontsize{11}{11}\selectfont{0}}} \\





\multicolumn{1}{>{\raggedright}m{\dimexpr 0.75in+0\tabcolsep}}{\textcolor[HTML]{000000}{\fontsize{11}{11}\selectfont{filiere\_4}}} & \multicolumn{1}{>{\raggedleft}m{\dimexpr 0.75in+0\tabcolsep}}{\textcolor[HTML]{000000}{\fontsize{11}{11}\selectfont{0}}} \\





\multicolumn{1}{>{\raggedright}m{\dimexpr 0.75in+0\tabcolsep}}{\textcolor[HTML]{000000}{\fontsize{11}{11}\selectfont{q8}}} & \multicolumn{1}{>{\raggedleft}m{\dimexpr 0.75in+0\tabcolsep}}{\textcolor[HTML]{000000}{\fontsize{11}{11}\selectfont{0}}} \\





\multicolumn{1}{>{\raggedright}m{\dimexpr 0.75in+0\tabcolsep}}{\textcolor[HTML]{000000}{\fontsize{11}{11}\selectfont{q81}}} & \multicolumn{1}{>{\raggedleft}m{\dimexpr 0.75in+0\tabcolsep}}{\textcolor[HTML]{000000}{\fontsize{11}{11}\selectfont{0}}} \\





\multicolumn{1}{>{\raggedright}m{\dimexpr 0.75in+0\tabcolsep}}{\textcolor[HTML]{000000}{\fontsize{11}{11}\selectfont{gps\_menlatitude}}} & \multicolumn{1}{>{\raggedleft}m{\dimexpr 0.75in+0\tabcolsep}}{\textcolor[HTML]{000000}{\fontsize{11}{11}\selectfont{0}}} \\





\multicolumn{1}{>{\raggedright}m{\dimexpr 0.75in+0\tabcolsep}}{\textcolor[HTML]{000000}{\fontsize{11}{11}\selectfont{gps\_menlongitude}}} & \multicolumn{1}{>{\raggedleft}m{\dimexpr 0.75in+0\tabcolsep}}{\textcolor[HTML]{000000}{\fontsize{11}{11}\selectfont{0}}} \\





\multicolumn{1}{>{\raggedright}m{\dimexpr 0.75in+0\tabcolsep}}{\textcolor[HTML]{000000}{\fontsize{11}{11}\selectfont{submissiondate}}} & \multicolumn{1}{>{\raggedleft}m{\dimexpr 0.75in+0\tabcolsep}}{\textcolor[HTML]{000000}{\fontsize{11}{11}\selectfont{0}}} \\





\multicolumn{1}{>{\raggedright}m{\dimexpr 0.75in+0\tabcolsep}}{\textcolor[HTML]{000000}{\fontsize{11}{11}\selectfont{start}}} & \multicolumn{1}{>{\raggedleft}m{\dimexpr 0.75in+0\tabcolsep}}{\textcolor[HTML]{000000}{\fontsize{11}{11}\selectfont{0}}} \\





\multicolumn{1}{>{\raggedright}m{\dimexpr 0.75in+0\tabcolsep}}{\textcolor[HTML]{000000}{\fontsize{11}{11}\selectfont{today}}} & \multicolumn{1}{>{\raggedleft}m{\dimexpr 0.75in+0\tabcolsep}}{\textcolor[HTML]{000000}{\fontsize{11}{11}\selectfont{0}}} \\

\ascline{1.5pt}{666666}{1-2}



\end{longtable}



\arrayrulecolor[HTML]{000000}

\global\setlength{\arrayrulewidth}{\Oldarrayrulewidth}

\global\setlength{\tabcolsep}{\Oldtabcolsep}

\renewcommand*{\arraystretch}{1}

\begin{Shaded}
\begin{Highlighting}[]
\CommentTok{\#apply(projet, MARGIN = 2, function(x) sum(is.na(x)) calcule les valeurs manquantes}
\CommentTok{\#par variables et flextable vient associer au nom de chaque variable sa valeur manquante.}
\end{Highlighting}
\end{Shaded}

\hypertarget{vuxe9rification-de-la-contenance-pour-la-variable-key-de-donnuxe9es-manquantes-dans-la-base-projet-et-identification-le-cas-uxe9chuxe9ant-du-statut-juridique-de-la-ou-des-pme-concernuxe9es.}{%
\paragraph{\texorpdfstring{Vérification de la contenance pour la
variable \textbf{key} de données manquantes dans la base \textbf{projet}
et identification le cas échéant du statut juridique de la (ou des) PME
concernées.}{Vérification de la contenance pour la variable key de données manquantes dans la base projet et identification le cas échéant du statut juridique de la (ou des) PME concernées.}}\label{vuxe9rification-de-la-contenance-pour-la-variable-key-de-donnuxe9es-manquantes-dans-la-base-projet-et-identification-le-cas-uxe9chuxe9ant-du-statut-juridique-de-la-ou-des-pme-concernuxe9es.}}

\begin{Shaded}
\begin{Highlighting}[]
\ControlFlowTok{if}\NormalTok{ (}\FunctionTok{all}\NormalTok{(}\SpecialCharTok{!}\FunctionTok{is.na}\NormalTok{(projet}\SpecialCharTok{$}\NormalTok{key))) \{}
  \FunctionTok{cat}\NormalTok{(}\StringTok{"La variable Key ne possède pas de valeurs manquantes"}\NormalTok{)}
\NormalTok{\} }\ControlFlowTok{else}\NormalTok{ \{}
  \FunctionTok{cat}\NormalTok{(}\FunctionTok{paste}\NormalTok{(}\StringTok{"Le nombre de valeurs manquantes de la variable key est"}\NormalTok{, }
              \FunctionTok{sum}\NormalTok{(}\FunctionTok{is.na}\NormalTok{(projet}\SpecialCharTok{$}\NormalTok{key))))}
  \FunctionTok{cat}\NormalTok{(}\StringTok{"}\SpecialCharTok{\textbackslash{}n}\StringTok{ Les types de PME concernées sont : }\SpecialCharTok{\textbackslash{}n}\StringTok{"}\NormalTok{, }
      \FunctionTok{paste}\NormalTok{(projet}\SpecialCharTok{$}\NormalTok{q12[}\FunctionTok{is.na}\NormalTok{(projet}\SpecialCharTok{$}\NormalTok{key)]), }\AttributeTok{sep=}\StringTok{" "}\NormalTok{)}
\NormalTok{\}}
\end{Highlighting}
\end{Shaded}

\begin{verbatim}
## La variable Key ne possède pas de valeurs manquantes
\end{verbatim}

\hypertarget{i.3-cruxe9ation-de-variables}{%
\subsubsection{\texorpdfstring{\textbf{I.3 Création de
variables}}{I.3 Création de variables}}\label{i.3-cruxe9ation-de-variables}}

\hypertarget{renommons-la-variable-q1-en-region}{%
\paragraph{\texorpdfstring{Renommons la variable \textbf{q1} en
\textbf{region}}{Renommons la variable q1 en region}}\label{renommons-la-variable-q1-en-region}}

\begin{Shaded}
\begin{Highlighting}[]
\ControlFlowTok{if}\NormalTok{(}\FunctionTok{is.na}\NormalTok{(}\FunctionTok{match}\NormalTok{(}\StringTok{"q1"}\NormalTok{, }\FunctionTok{colnames}\NormalTok{(projet))))\{}
  \FunctionTok{cat}\NormalTok{(}\StringTok{"la base \textquotesingle{}Projet\textquotesingle{} ne contient pas la variable \textquotesingle{}q1\textquotesingle{}"}\NormalTok{)}
\NormalTok{\}}\ControlFlowTok{else}\NormalTok{\{}
\NormalTok{  projet }\OtherTok{\textless{}{-}}\NormalTok{ projet }\SpecialCharTok{\%\textgreater{}\%}
  \FunctionTok{rename}\NormalTok{(}\AttributeTok{region =}\NormalTok{ q1)}
  \FunctionTok{cat}\NormalTok{(}\StringTok{"Variable \textquotesingle{}q1\textquotesingle{} renommée en \textquotesingle{}région\textquotesingle{}"}\NormalTok{)}
\NormalTok{\}}
\end{Highlighting}
\end{Shaded}

\begin{verbatim}
## Variable 'q1' renommée en 'région'
\end{verbatim}

\hypertarget{renommons-la-variable-q2-en-departement}{%
\paragraph{\texorpdfstring{Renommons la variable \textbf{q2} en
\textbf{departement}}{Renommons la variable q2 en departement}}\label{renommons-la-variable-q2-en-departement}}

\begin{Shaded}
\begin{Highlighting}[]
\ControlFlowTok{if}\NormalTok{(}\FunctionTok{is.na}\NormalTok{(}\FunctionTok{match}\NormalTok{(}\StringTok{"q2"}\NormalTok{, }\FunctionTok{colnames}\NormalTok{(projet))))\{}
  \FunctionTok{print}\NormalTok{(}\StringTok{"la base \textquotesingle{}Projet\textquotesingle{} ne contient pas la variable \textquotesingle{}q2\textquotesingle{}"}\NormalTok{)}
\NormalTok{\}}\ControlFlowTok{else}\NormalTok{\{}
\NormalTok{  projet }\OtherTok{\textless{}{-}}\NormalTok{ projet }\SpecialCharTok{\%\textgreater{}\%}
  \FunctionTok{rename}\NormalTok{(}\AttributeTok{departement =}\NormalTok{ q2)}
  \FunctionTok{print}\NormalTok{(}\StringTok{"Variable \textquotesingle{}q2\textquotesingle{} renommée en \textquotesingle{}département\textquotesingle{}"}\NormalTok{)}
\NormalTok{\}}
\end{Highlighting}
\end{Shaded}

\begin{verbatim}
## [1] "Variable 'q2' renommée en 'département'"
\end{verbatim}

\hypertarget{renommons-la-variable-q23-en-sexe}{%
\paragraph{\texorpdfstring{Renommons la variable \textbf{q23} en
\textbf{sexe}}{Renommons la variable q23 en sexe}}\label{renommons-la-variable-q23-en-sexe}}

\begin{Shaded}
\begin{Highlighting}[]
\ControlFlowTok{if}\NormalTok{(}\FunctionTok{is.na}\NormalTok{(}\FunctionTok{match}\NormalTok{(}\StringTok{"q23"}\NormalTok{, }\FunctionTok{colnames}\NormalTok{(projet))))\{}
  \FunctionTok{cat}\NormalTok{(}\StringTok{"la base \textquotesingle{}Projet\textquotesingle{} ne contient pas la variable \textquotesingle{}q23\textquotesingle{}"}\NormalTok{)}
\NormalTok{\}}\ControlFlowTok{else}\NormalTok{\{}
\NormalTok{  projet }\OtherTok{\textless{}{-}}\NormalTok{ projet }\SpecialCharTok{\%\textgreater{}\%}
  \FunctionTok{rename}\NormalTok{(}\AttributeTok{sexe =}\NormalTok{ q23)}
  \FunctionTok{cat}\NormalTok{(}\StringTok{"Variable \textquotesingle{}q23\textquotesingle{} renommée en \textquotesingle{}sexe\textquotesingle{}"}\NormalTok{)}
\NormalTok{\}}
\end{Highlighting}
\end{Shaded}

\begin{verbatim}
## Variable 'q23' renommée en 'sexe'
\end{verbatim}

\hypertarget{cruxe9ation-de-la-variable-sexe_2-qui-vaut-1-si-sexe-uxe9gale-uxe0-femme-et-0-sinon.}{%
\paragraph{\texorpdfstring{Création de la variable \textbf{sexe\_2} qui
vaut \textbf{1} si \textbf{sexe égale à Femme} et \textbf{0}
sinon.}{Création de la variable sexe\_2 qui vaut 1 si sexe égale à Femme et 0 sinon.}}\label{cruxe9ation-de-la-variable-sexe_2-qui-vaut-1-si-sexe-uxe9gale-uxe0-femme-et-0-sinon.}}

\begin{Shaded}
\begin{Highlighting}[]
\CommentTok{\#code de questionr::irec()}
\CommentTok{\# projet$sexe\_2 \textless{}{-} projet$sexe}
\CommentTok{\# projet$sexe\_2[projet$sexe == "Femme"] \textless{}{-} "1"}
\CommentTok{\# projet$sexe\_2[projet$sexe == "Homme"] \textless{}{-} "0"}
\CommentTok{\# projet$sexe\_2 \textless{}{-} as.numeric(projet$sexe\_2)}
\ControlFlowTok{if}\NormalTok{(}\FunctionTok{is.na}\NormalTok{(}\FunctionTok{match}\NormalTok{(}\StringTok{"sexe"}\NormalTok{,}\FunctionTok{colnames}\NormalTok{(projet))))\{}
  \FunctionTok{cat}\NormalTok{(}\StringTok{" la base \textquotesingle{}projet\textquotesingle{} ne contient pas la variable \textquotesingle{}sexe\textquotesingle{}"}\NormalTok{)}
\NormalTok{\}}\ControlFlowTok{else}\NormalTok{\{}
\NormalTok{  projet}\SpecialCharTok{$}\NormalTok{sexe\_2 }\OtherTok{\textless{}{-}} \FunctionTok{ifelse}\NormalTok{(projet}\SpecialCharTok{$}\NormalTok{sexe }\SpecialCharTok{==} \StringTok{"Femme"}\NormalTok{, }\DecValTok{1}\NormalTok{, }\DecValTok{0}\NormalTok{)}
  \FunctionTok{cat}\NormalTok{(}\StringTok{"Création de la variable sexe éffectué avec succès."}\NormalTok{)}
\NormalTok{\}}
\end{Highlighting}
\end{Shaded}

\begin{verbatim}
## Création de la variable sexe éffectué avec succès.
\end{verbatim}

\hypertarget{cruxe9ation-dun-data.frame-nommuxe9-langues-qui-prend-les-variables-key-et-les-variables-correspondantes-duxe9crites-plus-haut.}{%
\paragraph{\texorpdfstring{Création d'un \textbf{data.frame} nommé
\textbf{langues} qui prend les variables \textbf{key} et les variables
correspondantes décrites plus
haut.}{Création d'un data.frame nommé langues qui prend les variables key et les variables correspondantes décrites plus haut.}}\label{cruxe9ation-dun-data.frame-nommuxe9-langues-qui-prend-les-variables-key-et-les-variables-correspondantes-duxe9crites-plus-haut.}}

\begin{Shaded}
\begin{Highlighting}[]
\NormalTok{langues}\OtherTok{\textless{}{-}}\NormalTok{projet[}\FunctionTok{c}\NormalTok{(}\StringTok{"key"}\NormalTok{,}\FunctionTok{grep}\NormalTok{(}\StringTok{"\^{}q24a\_"}\NormalTok{, }\FunctionTok{colnames}\NormalTok{(projet), }\AttributeTok{value =} \ConstantTok{TRUE}\NormalTok{))]}
\CommentTok{\#la fonction grep() va aller lire dans les variables du dataframe projet}
\CommentTok{\# les variables dont le nom commence par "q24a\_" et l\textquotesingle{}on y ajoute la variable }
\CommentTok{\#key avec le c() et enfin, on passe en paramètre le resultat au dataframe projet}
\CommentTok{\# Cela permet de créer un nouveau dataframe "langues" avec les colonnes d\textquotesingle{}intérêt}
\end{Highlighting}
\end{Shaded}

\hypertarget{cruxe9ation-dune-variable-parle-qui-est-uxe9gale-au-nombre-de-langue-parluxe9e-par-le-dirigeant-de-la-pme.}{%
\paragraph{\texorpdfstring{Création d'une variable \textbf{parle} qui
est égale au nombre de langue parlée par le dirigeant de la
PME.}{Création d'une variable parle qui est égale au nombre de langue parlée par le dirigeant de la PME.}}\label{cruxe9ation-dune-variable-parle-qui-est-uxe9gale-au-nombre-de-langue-parluxe9e-par-le-dirigeant-de-la-pme.}}

\begin{Shaded}
\begin{Highlighting}[]
\CommentTok{\# La nouvelle colonne contient la somme des valeurs de chaque ligne pour les}
\CommentTok{\#colonnes qui commencent par "q24a\_" dans le dataframe "langues"}
\NormalTok{langues}\SpecialCharTok{$}\NormalTok{parle }\OtherTok{\textless{}{-}} \FunctionTok{rowSums}\NormalTok{(langues[, }\FunctionTok{grepl}\NormalTok{(}\StringTok{"\^{}q24a\_"}\NormalTok{, }\FunctionTok{colnames}\NormalTok{(langues))])}
\end{Highlighting}
\end{Shaded}

\hypertarget{suxe9lection-unique-des-variables-key-et-parle-et-affectation-au-dataframe-langues.}{%
\paragraph{\texorpdfstring{Sélection unique des variables \textbf{key}
et \textbf{parle}, et affectation au dataframe
\textbf{langues}.}{Sélection unique des variables key et parle, et affectation au dataframe langues.}}\label{suxe9lection-unique-des-variables-key-et-parle-et-affectation-au-dataframe-langues.}}

\begin{Shaded}
\begin{Highlighting}[]
\CommentTok{\#ici on va écraser le dataframe langue en conservant uniquement les deux}
\CommentTok{\# variables key et parle}
\NormalTok{langues}\OtherTok{\textless{}{-}}\NormalTok{langues[,}\FunctionTok{c}\NormalTok{(}\StringTok{"key"}\NormalTok{, }\StringTok{"parle"}\NormalTok{)]}
\end{Highlighting}
\end{Shaded}

\hypertarget{merge-des-data.frame-projet-et-langues}{%
\paragraph{\texorpdfstring{Merge des \textbf{data.frame} \textbf{projet}
et
\textbf{langues}}{Merge des data.frame projet et langues}}\label{merge-des-data.frame-projet-et-langues}}

\begin{Shaded}
\begin{Highlighting}[]
\NormalTok{projet}\OtherTok{\textless{}{-}}\NormalTok{base}\SpecialCharTok{::}\FunctionTok{merge}\NormalTok{(projet, langues, }\AttributeTok{by =} \StringTok{"key"}\NormalTok{, }\AttributeTok{all =} \ConstantTok{TRUE}\NormalTok{)}
\end{Highlighting}
\end{Shaded}

\hypertarget{ii.-analyses-descriptives}{%
\subsection{II. Analyses descriptives}\label{ii.-analyses-descriptives}}

\hypertarget{ruxe9partion-des-pme-suivant-le-sexe.}{%
\subsubsection{Répartion des PME suivant le
sexe.}\label{ruxe9partion-des-pme-suivant-le-sexe.}}

\begin{Shaded}
\begin{Highlighting}[]
\NormalTok{gtsummary}\SpecialCharTok{::}\FunctionTok{tbl\_summary}\NormalTok{(projet, }\AttributeTok{include=}\NormalTok{sexe) }\SpecialCharTok{\%\textgreater{}\%} \FunctionTok{modify\_header}\NormalTok{(label}\SpecialCharTok{\textasciitilde{}}\StringTok{"**Total**"}
\NormalTok{                                                    ) }\SpecialCharTok{\%\textgreater{}\%} \FunctionTok{modify\_caption}\NormalTok{(}
                                                      \AttributeTok{caption =} \StringTok{"**Répartition des PME suivant le sexe**"}\NormalTok{)}
\end{Highlighting}
\end{Shaded}

\begin{longtable}[]{@{}lc@{}}
\caption{\textbf{Répartition des PME suivant le sexe}}\tabularnewline
\toprule\noalign{}
\textbf{Total} & \textbf{N = 250} \\
\midrule\noalign{}
\endfirsthead
\toprule\noalign{}
\textbf{Total} & \textbf{N = 250} \\
\midrule\noalign{}
\endhead
\bottomrule\noalign{}
\endlastfoot
sexe & \\
Femme & 191 (76\%) \\
Homme & 59 (24\%) \\
\end{longtable}

\hypertarget{ruxe9partion-des-pme-suivant-le-niveau-dinstruction.}{%
\subsubsection{Répartion des PME suivant le niveau
d'instruction.}\label{ruxe9partion-des-pme-suivant-le-niveau-dinstruction.}}

\begin{Shaded}
\begin{Highlighting}[]
\NormalTok{gtsummary}\SpecialCharTok{::}\FunctionTok{tbl\_summary}\NormalTok{(projet, }\AttributeTok{include=}\NormalTok{q25, }\AttributeTok{label =} \FunctionTok{list}\NormalTok{(q25}\SpecialCharTok{\textasciitilde{}}\StringTok{"Niveau d\textquotesingle{}instruction"}\NormalTok{)}
\NormalTok{            ) }\SpecialCharTok{\%\textgreater{}\%} \FunctionTok{modify\_header}\NormalTok{(label}\SpecialCharTok{\textasciitilde{}}\StringTok{"**Total**"}\NormalTok{) }\SpecialCharTok{\%\textgreater{}\%} \FunctionTok{modify\_caption}\NormalTok{(}
              \AttributeTok{caption =} \StringTok{"**Répartition des PME suivant }
\StringTok{              le niveau d’instruction**"}\NormalTok{)}
\end{Highlighting}
\end{Shaded}

\begin{longtable}[]{@{}lc@{}}
\caption{\textbf{Répartition des PME suivant le niveau
d'instruction}}\tabularnewline
\toprule\noalign{}
\textbf{Total} & \textbf{N = 250} \\
\midrule\noalign{}
\endfirsthead
\toprule\noalign{}
\textbf{Total} & \textbf{N = 250} \\
\midrule\noalign{}
\endhead
\bottomrule\noalign{}
\endlastfoot
Niveau d'instruction & \\
Aucun niveau & 79 (32\%) \\
Niveau primaire & 56 (22\%) \\
Niveau secondaire & 74 (30\%) \\
Niveau Superieur & 41 (16\%) \\
\end{longtable}

\hypertarget{ruxe9partion-des-pme-suivant-le-statut-juridique.}{%
\subsubsection{Répartion des PME suivant le statut
juridique.}\label{ruxe9partion-des-pme-suivant-le-statut-juridique.}}

\begin{Shaded}
\begin{Highlighting}[]
\NormalTok{gtsummary}\SpecialCharTok{::}\FunctionTok{tbl\_summary}\NormalTok{(projet, }\AttributeTok{include=}\NormalTok{q12, }\AttributeTok{label =} \FunctionTok{list}\NormalTok{(q12}\SpecialCharTok{\textasciitilde{}}\StringTok{"Statut juridique"}\NormalTok{)}
\NormalTok{            ) }\SpecialCharTok{\%\textgreater{}\%} \FunctionTok{modify\_header}\NormalTok{(label}\SpecialCharTok{\textasciitilde{}}\StringTok{"**Total**"}\NormalTok{) }\SpecialCharTok{\%\textgreater{}\%} \FunctionTok{modify\_caption}\NormalTok{(}
              \AttributeTok{caption =} \StringTok{"**Répartition des PME suivant}
\StringTok{              le statut juridique**"}\NormalTok{)}
\end{Highlighting}
\end{Shaded}

\begin{longtable}[]{@{}lc@{}}
\caption{\textbf{Répartition des PME suivant le statut
juridique}}\tabularnewline
\toprule\noalign{}
\textbf{Total} & \textbf{N = 250} \\
\midrule\noalign{}
\endfirsthead
\toprule\noalign{}
\textbf{Total} & \textbf{N = 250} \\
\midrule\noalign{}
\endhead
\bottomrule\noalign{}
\endlastfoot
Statut juridique & \\
Association & 6 (2.4\%) \\
GIE & 179 (72\%) \\
Informel & 38 (15\%) \\
SA & 7 (2.8\%) \\
SARL & 13 (5.2\%) \\
SUARL & 7 (2.8\%) \\
\end{longtable}

\hypertarget{ruxe9partion-des-pme-suivant-le-propriuxe9tairelocataire.}{%
\subsubsection{Répartion des PME suivant le
propriétaire/locataire.}\label{ruxe9partion-des-pme-suivant-le-propriuxe9tairelocataire.}}

\begin{Shaded}
\begin{Highlighting}[]
\NormalTok{gtsummary}\SpecialCharTok{::}\FunctionTok{tbl\_summary}\NormalTok{(projet, }\AttributeTok{include=}\NormalTok{q81, }\AttributeTok{label =} \FunctionTok{list}\NormalTok{(q81}\SpecialCharTok{\textasciitilde{}}\StringTok{"propriétaire/locataire"}\NormalTok{)}
\NormalTok{            )}\SpecialCharTok{\%\textgreater{}\%} \FunctionTok{modify\_header}\NormalTok{(label}\SpecialCharTok{\textasciitilde{}}\StringTok{"**Total**"}
\NormalTok{                               ) }\SpecialCharTok{\%\textgreater{}\%} 
  \FunctionTok{modify\_caption}\NormalTok{(}\AttributeTok{caption =} \StringTok{"**Répartition }
\StringTok{                 des PME suivant le type de logement**"}\NormalTok{)}
\end{Highlighting}
\end{Shaded}

\begin{longtable}[]{@{}lc@{}}
\caption{\textbf{Répartition des PME suivant le type de
logement}}\tabularnewline
\toprule\noalign{}
\textbf{Total} & \textbf{N = 250} \\
\midrule\noalign{}
\endfirsthead
\toprule\noalign{}
\textbf{Total} & \textbf{N = 250} \\
\midrule\noalign{}
\endhead
\bottomrule\noalign{}
\endlastfoot
propriétaire/locataire & \\
Locataire & 24 (9.6\%) \\
Propriétaire & 226 (90\%) \\
\end{longtable}

\hypertarget{ruxe9partion-des-pme-suivant-le-statut-juridique-et-le-sexe.}{%
\subsubsection{Répartion des PME suivant le statut juridique et le
sexe.}\label{ruxe9partion-des-pme-suivant-le-statut-juridique-et-le-sexe.}}

\begin{Shaded}
\begin{Highlighting}[]
\NormalTok{gtsummary}\SpecialCharTok{::}\FunctionTok{tbl\_cross}\NormalTok{(projet,}
    \AttributeTok{row =}\NormalTok{  q12,}
    \AttributeTok{col =}\NormalTok{ sexe,}
    \AttributeTok{percent =} \StringTok{"row"}\NormalTok{,}
    \AttributeTok{label =} \FunctionTok{list}\NormalTok{(q12}\SpecialCharTok{\textasciitilde{}}\StringTok{"Statut juridique"}\NormalTok{,sexe}\SpecialCharTok{\textasciitilde{}}\StringTok{"sexe"}\NormalTok{)}
\NormalTok{  ) }\SpecialCharTok{\%\textgreater{}\%} \FunctionTok{modify\_caption}\NormalTok{(}\AttributeTok{caption =} 
                         \StringTok{"**Répartition des PME suivant}
\StringTok{                       le statut juridique et le sexe**"}\NormalTok{) }
\end{Highlighting}
\end{Shaded}

\begin{longtable}[]{@{}lccc@{}}
\caption{\textbf{Répartition des PME suivant le statut juridique et le
sexe}}\tabularnewline
\toprule\noalign{}
& Femme & Homme & Total \\
\midrule\noalign{}
\endfirsthead
\toprule\noalign{}
& Femme & Homme & Total \\
\midrule\noalign{}
\endhead
\bottomrule\noalign{}
\endlastfoot
Statut juridique & & & \\
Association & 3 (50\%) & 3 (50\%) & 6 (100\%) \\
GIE & 149 (83\%) & 30 (17\%) & 179 (100\%) \\
Informel & 32 (84\%) & 6 (16\%) & 38 (100\%) \\
SA & 1 (14\%) & 6 (86\%) & 7 (100\%) \\
SARL & 2 (15\%) & 11 (85\%) & 13 (100\%) \\
SUARL & 4 (57\%) & 3 (43\%) & 7 (100\%) \\
Total & 191 (76\%) & 59 (24\%) & 250 (100\%) \\
\end{longtable}

\hypertarget{ruxe9partion-des-pme-suivant-le-niveau-dinstruction-et-le-sexe.}{%
\subsubsection{Répartion des PME suivant le niveau d'instruction et le
sexe.}\label{ruxe9partion-des-pme-suivant-le-niveau-dinstruction-et-le-sexe.}}

\begin{Shaded}
\begin{Highlighting}[]
\NormalTok{gtsummary}\SpecialCharTok{::}\FunctionTok{tbl\_cross}\NormalTok{(projet,}
    \AttributeTok{row =}\NormalTok{  q25,}
    \AttributeTok{col =}\NormalTok{ sexe,}
    \AttributeTok{percent =} \StringTok{"row"}\NormalTok{,}
    \AttributeTok{label =} \FunctionTok{list}\NormalTok{(q25}\SpecialCharTok{\textasciitilde{}}\StringTok{"Niveau d\textquotesingle{}instruction"}\NormalTok{,sexe}\SpecialCharTok{\textasciitilde{}}\StringTok{"**sexe**"}\NormalTok{)}
\NormalTok{  ) }\SpecialCharTok{\%\textgreater{}\%} \FunctionTok{modify\_caption}\NormalTok{(}\AttributeTok{caption =} 
                         \StringTok{"**Répartition des PME suivant}
\StringTok{                       le niveau d\textquotesingle{}instruction et le sexe**"}\NormalTok{) }
\end{Highlighting}
\end{Shaded}

\begin{longtable}[]{@{}lccc@{}}
\caption{\textbf{Répartition des PME suivant le niveau d'instruction et
le sexe}}\tabularnewline
\toprule\noalign{}
& Femme & Homme & Total \\
\midrule\noalign{}
\endfirsthead
\toprule\noalign{}
& Femme & Homme & Total \\
\midrule\noalign{}
\endhead
\bottomrule\noalign{}
\endlastfoot
Niveau d'instruction & & & \\
Aucun niveau & 70 (89\%) & 9 (11\%) & 79 (100\%) \\
Niveau primaire & 48 (86\%) & 8 (14\%) & 56 (100\%) \\
Niveau secondaire & 56 (76\%) & 18 (24\%) & 74 (100\%) \\
Niveau Superieur & 17 (41\%) & 24 (59\%) & 41 (100\%) \\
Total & 191 (76\%) & 59 (24\%) & 250 (100\%) \\
\end{longtable}

\hypertarget{ruxe9partion-des-pme-suivant-le-propriuxe9tairelocataire-et-le-sexe.}{%
\subsubsection{Répartion des PME suivant le Propriétaire/locataire et le
sexe.}\label{ruxe9partion-des-pme-suivant-le-propriuxe9tairelocataire-et-le-sexe.}}

\begin{Shaded}
\begin{Highlighting}[]
\NormalTok{gtsummary}\SpecialCharTok{::}\FunctionTok{tbl\_cross}\NormalTok{(projet,}
    \AttributeTok{row =}\NormalTok{  q81,}
    \AttributeTok{col =}\NormalTok{ sexe,}
    \AttributeTok{percent =} \StringTok{"row"}\NormalTok{,}
    \AttributeTok{label =} \FunctionTok{list}\NormalTok{(q81}\SpecialCharTok{\textasciitilde{}}\StringTok{"Propriétaire/locataire"}\NormalTok{,sexe}\SpecialCharTok{\textasciitilde{}}\StringTok{"**sexe**"}\NormalTok{)}
\NormalTok{  ) }\SpecialCharTok{\%\textgreater{}\%} \FunctionTok{modify\_caption}\NormalTok{(}\AttributeTok{caption =} 
                         \StringTok{"**Répartition des PME suivant }
\StringTok{                       le niveau d\textquotesingle{}instruction et le sexe**"}\NormalTok{) }
\end{Highlighting}
\end{Shaded}

\begin{longtable}[]{@{}lccc@{}}
\caption{\textbf{Répartition des PME suivant le niveau d'instruction et
le sexe}}\tabularnewline
\toprule\noalign{}
& Femme & Homme & Total \\
\midrule\noalign{}
\endfirsthead
\toprule\noalign{}
& Femme & Homme & Total \\
\midrule\noalign{}
\endhead
\bottomrule\noalign{}
\endlastfoot
Propriétaire/locataire & & & \\
Locataire & 16 (67\%) & 8 (33\%) & 24 (100\%) \\
Propriétaire & 175 (77\%) & 51 (23\%) & 226 (100\%) \\
Total & 191 (76\%) & 59 (24\%) & 250 (100\%) \\
\end{longtable}

\hypertarget{statistiques-descriptives-de-notre-choix-sur-les-autres-variables-uxe0-duxe9finir.}{%
\subsubsection{\texorpdfstring{\textbf{Statistiques descriptives de
notre choix sur les autres variables (à
définir).}}{Statistiques descriptives de notre choix sur les autres variables (à définir).}}\label{statistiques-descriptives-de-notre-choix-sur-les-autres-variables-uxe0-duxe9finir.}}

\begin{Shaded}
\begin{Highlighting}[]
\CommentTok{\# Répartition des PME suivant le sexe, le niveau d\textquotesingle{}instruction, le statut juridique et le propriétaire locataire}

\NormalTok{tbl1 }\OtherTok{\textless{}{-}}\NormalTok{ projet }\SpecialCharTok{\%\textgreater{}\%}\NormalTok{ gtsummary}\SpecialCharTok{::}\FunctionTok{tbl\_summary}\NormalTok{(}\AttributeTok{include =} \FunctionTok{c}\NormalTok{(}\StringTok{"sexe"}\NormalTok{, }\StringTok{"q25"}\NormalTok{, }\StringTok{"q12"}\NormalTok{, }\StringTok{"q81"}\NormalTok{))}
\CommentTok{\# Répartition des PME suivant le statut juridique et le sexe,   le niveau d’instruction et le sexe, •   Propriétaire/locataire suivant le sexe}
\NormalTok{tbl2 }\OtherTok{\textless{}{-}}\NormalTok{ projet }\SpecialCharTok{\%\textgreater{}\%} \FunctionTok{tbl\_summary}\NormalTok{(}
  \AttributeTok{include =} \FunctionTok{c}\NormalTok{(}\StringTok{"q25"}\NormalTok{, }\StringTok{"q12"}\NormalTok{, }\StringTok{"q81"}\NormalTok{), }
  \AttributeTok{by =}\StringTok{"sexe"}\NormalTok{, }\AttributeTok{label=}\FunctionTok{list}\NormalTok{(q12}\SpecialCharTok{\textasciitilde{}} \StringTok{"Statut juridique"}\NormalTok{, }
\NormalTok{                         q25}\SpecialCharTok{\textasciitilde{}} \StringTok{"Niveau d\textquotesingle{}instruction"}\NormalTok{, }
\NormalTok{                         q81}\SpecialCharTok{\textasciitilde{}} \StringTok{"Propriétaire/locataire"}\NormalTok{)) }\SpecialCharTok{\%\textgreater{}\%} 
  \FunctionTok{add\_overall}\NormalTok{() }\SpecialCharTok{\%\textgreater{}\%} 
  \FunctionTok{modify\_header}\NormalTok{(label }\SpecialCharTok{\textasciitilde{}} \StringTok{"**Caractéristiques**"}\NormalTok{) }\SpecialCharTok{\%\textgreater{}\%}
  \FunctionTok{modify\_spanning\_header}\NormalTok{(}\FunctionTok{c}\NormalTok{(}\StringTok{"stat\_1"}\NormalTok{, }\StringTok{"stat\_2"}\NormalTok{) }\SpecialCharTok{\textasciitilde{}} \StringTok{"**Sexe**"}\NormalTok{) }

\DocumentationTok{\#\# Empillement l\textquotesingle{}un sur l\textquotesingle{}autre}

\NormalTok{gtsummary}\SpecialCharTok{::}\FunctionTok{tbl\_stack}\NormalTok{(}
  \FunctionTok{list}\NormalTok{(tbl1, tbl2),}
  \AttributeTok{group\_header =} \FunctionTok{c}\NormalTok{(}\StringTok{"Modèle univarié"}\NormalTok{, }\StringTok{"Modèle bivaré"}\NormalTok{) }\DocumentationTok{\#\# intitulé des groupes de tableau associés}
\NormalTok{)}
\end{Highlighting}
\end{Shaded}

\begin{longtable}[]{@{}
  >{\raggedright\arraybackslash}p{(\columnwidth - 8\tabcolsep) * \real{0.1758}}
  >{\raggedright\arraybackslash}p{(\columnwidth - 8\tabcolsep) * \real{0.2527}}
  >{\centering\arraybackslash}p{(\columnwidth - 8\tabcolsep) * \real{0.1429}}
  >{\centering\arraybackslash}p{(\columnwidth - 8\tabcolsep) * \real{0.2198}}
  >{\centering\arraybackslash}p{(\columnwidth - 8\tabcolsep) * \real{0.2088}}@{}}
\toprule\noalign{}
\begin{minipage}[b]{\linewidth}\raggedright
\textbf{Group}
\end{minipage} & \begin{minipage}[b]{\linewidth}\raggedright
\textbf{Characteristic}
\end{minipage} & \begin{minipage}[b]{\linewidth}\centering
\textbf{N = 250}
\end{minipage} & \begin{minipage}[b]{\linewidth}\centering
\textbf{Femme}, N = 191
\end{minipage} & \begin{minipage}[b]{\linewidth}\centering
\textbf{Homme}, N = 59
\end{minipage} \\
\midrule\noalign{}
\endhead
\bottomrule\noalign{}
\endlastfoot
Modèle univarié & sexe & & & \\
& Femme & 191 (76\%) & & \\
& Homme & 59 (24\%) & & \\
& q25 & & & \\
& Aucun niveau & 79 (32\%) & & \\
& Niveau primaire & 56 (22\%) & & \\
& Niveau secondaire & 74 (30\%) & & \\
& Niveau Superieur & 41 (16\%) & & \\
& q12 & & & \\
& Association & 6 (2.4\%) & & \\
& GIE & 179 (72\%) & & \\
& Informel & 38 (15\%) & & \\
& SA & 7 (2.8\%) & & \\
& SARL & 13 (5.2\%) & & \\
& SUARL & 7 (2.8\%) & & \\
& q81 & & & \\
& Locataire & 24 (9.6\%) & & \\
& Propriétaire & 226 (90\%) & & \\
Modèle bivaré & Niveau d'instruction & & & \\
& Aucun niveau & 79 (32\%) & 70 (37\%) & 9 (15\%) \\
& Niveau primaire & 56 (22\%) & 48 (25\%) & 8 (14\%) \\
& Niveau secondaire & 74 (30\%) & 56 (29\%) & 18 (31\%) \\
& Niveau Superieur & 41 (16\%) & 17 (8.9\%) & 24 (41\%) \\
& Statut juridique & & & \\
& Association & 6 (2.4\%) & 3 (1.6\%) & 3 (5.1\%) \\
& GIE & 179 (72\%) & 149 (78\%) & 30 (51\%) \\
& Informel & 38 (15\%) & 32 (17\%) & 6 (10\%) \\
& SA & 7 (2.8\%) & 1 (0.5\%) & 6 (10\%) \\
& SARL & 13 (5.2\%) & 2 (1.0\%) & 11 (19\%) \\
& SUARL & 7 (2.8\%) & 4 (2.1\%) & 3 (5.1\%) \\
& Propriétaire/locataire & & & \\
& Locataire & 24 (9.6\%) & 16 (8.4\%) & 8 (14\%) \\
& Propriétaire & 226 (90\%) & 175 (92\%) & 51 (86\%) \\
\end{longtable}

\begin{Shaded}
\begin{Highlighting}[]
\DocumentationTok{\#\# Agencement l\textquotesingle{}une à coté de l\textquotesingle{}autre}
\NormalTok{gtsummary}\SpecialCharTok{::}\FunctionTok{tbl\_merge}\NormalTok{(}
  \FunctionTok{list}\NormalTok{(tbl1, tbl2),}
  \AttributeTok{tab\_spanner =} \FunctionTok{c}\NormalTok{(}\StringTok{"Modèle bivarié"}\NormalTok{, }\StringTok{"Modèle multivarié"}\NormalTok{) }\DocumentationTok{\#\# intitulé des groupes de tableau associés}
\NormalTok{)}
\end{Highlighting}
\end{Shaded}

\begin{longtable}[]{@{}
  >{\raggedright\arraybackslash}p{(\columnwidth - 8\tabcolsep) * \real{0.2371}}
  >{\centering\arraybackslash}p{(\columnwidth - 8\tabcolsep) * \real{0.1340}}
  >{\centering\arraybackslash}p{(\columnwidth - 8\tabcolsep) * \real{0.2268}}
  >{\centering\arraybackslash}p{(\columnwidth - 8\tabcolsep) * \real{0.2062}}
  >{\centering\arraybackslash}p{(\columnwidth - 8\tabcolsep) * \real{0.1959}}@{}}
\toprule\noalign{}
\begin{minipage}[b]{\linewidth}\raggedright
\textbf{Characteristic}
\end{minipage} & \begin{minipage}[b]{\linewidth}\centering
\textbf{N = 250}
\end{minipage} & \begin{minipage}[b]{\linewidth}\centering
\textbf{Overall}, N = 250
\end{minipage} & \begin{minipage}[b]{\linewidth}\centering
\textbf{Femme}, N = 191
\end{minipage} & \begin{minipage}[b]{\linewidth}\centering
\textbf{Homme}, N = 59
\end{minipage} \\
\midrule\noalign{}
\endhead
\bottomrule\noalign{}
\endlastfoot
sexe & & & & \\
Femme & 191 (76\%) & & & \\
Homme & 59 (24\%) & & & \\
q25 & & & & \\
Aucun niveau & 79 (32\%) & & & \\
Niveau primaire & 56 (22\%) & & & \\
Niveau secondaire & 74 (30\%) & & & \\
Niveau Superieur & 41 (16\%) & & & \\
q12 & & & & \\
Association & 6 (2.4\%) & & & \\
GIE & 179 (72\%) & & & \\
Informel & 38 (15\%) & & & \\
SA & 7 (2.8\%) & & & \\
SARL & 13 (5.2\%) & & & \\
SUARL & 7 (2.8\%) & & & \\
q81 & & & & \\
Locataire & 24 (9.6\%) & & & \\
Propriétaire & 226 (90\%) & & & \\
Niveau d'instruction & & & & \\
Aucun niveau & & 79 (32\%) & 70 (37\%) & 9 (15\%) \\
Niveau primaire & & 56 (22\%) & 48 (25\%) & 8 (14\%) \\
Niveau secondaire & & 74 (30\%) & 56 (29\%) & 18 (31\%) \\
Niveau Superieur & & 41 (16\%) & 17 (8.9\%) & 24 (41\%) \\
Statut juridique & & & & \\
Association & & 6 (2.4\%) & 3 (1.6\%) & 3 (5.1\%) \\
GIE & & 179 (72\%) & 149 (78\%) & 30 (51\%) \\
Informel & & 38 (15\%) & 32 (17\%) & 6 (10\%) \\
SA & & 7 (2.8\%) & 1 (0.5\%) & 6 (10\%) \\
SARL & & 13 (5.2\%) & 2 (1.0\%) & 11 (19\%) \\
SUARL & & 7 (2.8\%) & 4 (2.1\%) & 3 (5.1\%) \\
Propriétaire/locataire & & & & \\
Locataire & & 24 (9.6\%) & 16 (8.4\%) & 8 (14\%) \\
Propriétaire & & 226 (90\%) & 175 (92\%) & 51 (86\%) \\
\end{longtable}

\begin{Shaded}
\begin{Highlighting}[]
\CommentTok{\# \# Statistiques descriptives}
\CommentTok{\# }
\CommentTok{\# gtsummary::tbl\_stack(}
\CommentTok{\#   list(tbl\_filiere\_1 , tbl\_filiere\_2 , tbl\_filiere\_3 , tbl\_filiere\_4),}
\CommentTok{\#   group\_header = c("arachide", "anacarde", "mangue", "riz") \#\# intitulé des groupes de tableau associés}
\CommentTok{\# )}
\CommentTok{\# }
\CommentTok{\# \#\# Agencement l\textquotesingle{}une à coté de l\textquotesingle{}autre}
\CommentTok{\# gtsummary::tbl\_merge(}
\CommentTok{\#   list(tbl\_filiere\_1, tbl\_filiere\_2, tbl\_filiere\_3, tbl\_filiere\_4),}
\CommentTok{\#   tab\_spanner = c("arachide", "anacarde", "mangue", "riz") \#\# intitulé des groupes de tableau associés}
\CommentTok{\# )}
\end{Highlighting}
\end{Shaded}

\hypertarget{iii.-un-peu-de-cartographie}{%
\subsection{III. Un peu de
cartographie}\label{iii.-un-peu-de-cartographie}}

\hypertarget{transformation-du-data.frame-en-donnuxe9es-guxe9ographiques-dont-lobjet-sera-nommuxe9-projet_map.}{%
\subsubsection{\texorpdfstring{Transformation du \textbf{data.frame} en
données géographiques dont l'objet sera nommé
\textbf{projet\_map}.}{Transformation du data.frame en données géographiques dont l'objet sera nommé projet\_map.}}\label{transformation-du-data.frame-en-donnuxe9es-guxe9ographiques-dont-lobjet-sera-nommuxe9-projet_map.}}

\begin{Shaded}
\begin{Highlighting}[]
\CommentTok{\# Jointure spatiale entre les données du projet et les données géospatiales du Sénégal}
\CommentTok{\# 1. Lecture des données géospatiales du Sénégal avec la fonction st\_read()}
\CommentTok{\# et spécification du système de coordonnées de référence (CRS)}
\CommentTok{\#senegal \textless{}{-} st\_read("gadm41\_SEN\_1.shp")}
\CommentTok{\# 2. Conversion des données du projet en un objet spatial "sf"}
\CommentTok{\# en utilisant la fonction st\_as\_sf()}
\CommentTok{\# {-} Les colonnes "gps\_menlongitude" et "gps\_menlatitude" contiennent les coordonnées spatiales}
\CommentTok{\# {-} Le CRS est spécifié à l\textquotesingle{}aide de la fonction st\_crs() pour correspondre au CRS des données du Sénégal}
\CommentTok{\# {-} Les données projet sont transformées en objet spatial "sf" avec st\_as\_sf()}
\CommentTok{\# projet\_sf \textless{}{-} st\_as\_sf(projet, }
\CommentTok{\#                       coords = c("gps\_menlongitude", "gps\_menlatitude"), }
\CommentTok{\#                       crs = st\_crs(senegal))}
\CommentTok{\# 3. Jointure spatiale entre les objets spatiaux "projet\_sf" et "senegal"}
\CommentTok{\# en utilisant la fonction st\_join()}
\CommentTok{\# {-} La fonction st\_join() effectue la jointure spatiale entre les polygones du Sénégal (senegal)}
\CommentTok{\#   et les points du projet (projet\_sf) en attribuant à chaque point son emplacement spatial}
\CommentTok{\#   en fonction de la région du Sénégal dans laquelle il se trouve}
\CommentTok{\#code final}
\NormalTok{projet\_map }\OtherTok{\textless{}{-}} \FunctionTok{st\_join}\NormalTok{(}\FunctionTok{st\_as\_sf}\NormalTok{(projet, }
                      \AttributeTok{coords =} \FunctionTok{c}\NormalTok{(}\StringTok{"gps\_menlongitude"}\NormalTok{, }\StringTok{"gps\_menlatitude"}\NormalTok{), }
                      \AttributeTok{crs=}\FunctionTok{st\_crs}\NormalTok{(}\FunctionTok{st\_read}\NormalTok{(}\StringTok{"gadm41\_SEN\_1.shp"}\NormalTok{))),}
                      \FunctionTok{st\_read}\NormalTok{(}\StringTok{"gadm41\_SEN\_1.shp"}\NormalTok{))}
\end{Highlighting}
\end{Shaded}

\begin{verbatim}
## Reading layer `gadm41_SEN_1' from data source 
##   `C:\Users\starlab\Desktop\Projet_R\gadm41_SEN_1.shp' using driver `ESRI Shapefile'
## Simple feature collection with 14 features and 11 fields
## Geometry type: MULTIPOLYGON
## Dimension:     XY
## Bounding box:  xmin: -17.54319 ymin: 12.30786 xmax: -11.34247 ymax: 16.69207
## Geodetic CRS:  WGS 84
## Reading layer `gadm41_SEN_1' from data source 
##   `C:\Users\starlab\Desktop\Projet_R\gadm41_SEN_1.shp' using driver `ESRI Shapefile'
## Simple feature collection with 14 features and 11 fields
## Geometry type: MULTIPOLYGON
## Dimension:     XY
## Bounding box:  xmin: -17.54319 ymin: 12.30786 xmax: -11.34247 ymax: 16.69207
## Geodetic CRS:  WGS 84
\end{verbatim}

\hypertarget{repruxe9sentation-spatiale-des-pme-suivant-le-sexe.}{%
\subsubsection{Représentation spatiale des PME suivant le
sexe.}\label{repruxe9sentation-spatiale-des-pme-suivant-le-sexe.}}

\begin{Shaded}
\begin{Highlighting}[]
\FunctionTok{ggplot}\NormalTok{() }\SpecialCharTok{+}
  \FunctionTok{geom\_sf}\NormalTok{(}\AttributeTok{data=}\FunctionTok{st\_read}\NormalTok{(}\StringTok{"gadm41\_SEN\_1.shp"}\NormalTok{))}\SpecialCharTok{+}
  \FunctionTok{geom\_sf\_text}\NormalTok{(}\AttributeTok{data=}\FunctionTok{st\_read}\NormalTok{(}\StringTok{"gadm41\_SEN\_1.shp"}\NormalTok{), }\FunctionTok{aes}\NormalTok{(}\AttributeTok{label=}\NormalTok{NAME\_1))}\SpecialCharTok{+}
  \FunctionTok{geom\_sf}\NormalTok{(}\AttributeTok{data=}\NormalTok{projet\_map, }\FunctionTok{aes}\NormalTok{(}\AttributeTok{color=}\NormalTok{sexe), }\AttributeTok{size=}\FloatTok{1.5}\NormalTok{)}\SpecialCharTok{+}
  \FunctionTok{labs}\NormalTok{(}\AttributeTok{title =} \StringTok{"Repartition des PME suivant le sexe"}\NormalTok{,}
       \AttributeTok{subtitle =} \StringTok{"Carte du Sénégal"}\NormalTok{,}
       \AttributeTok{color =} \StringTok{"Sexe"}\NormalTok{, }\AttributeTok{x =} \ConstantTok{NULL}\NormalTok{, }\AttributeTok{y =} \ConstantTok{NULL}\NormalTok{) }\SpecialCharTok{+}
  \FunctionTok{theme\_minimal}\NormalTok{() }\SpecialCharTok{+}
  \FunctionTok{theme}\NormalTok{(}
    \AttributeTok{plot.title =} \FunctionTok{element\_text}\NormalTok{(}\AttributeTok{hjust =} \DecValTok{1}\NormalTok{),}
    \AttributeTok{plot.subtitle =} \FunctionTok{element\_text}\NormalTok{(}\AttributeTok{hjust =} \DecValTok{1}\NormalTok{))}\SpecialCharTok{+}
  \FunctionTok{theme\_void}\NormalTok{()}
\end{Highlighting}
\end{Shaded}

\begin{verbatim}
## Reading layer `gadm41_SEN_1' from data source 
##   `C:\Users\starlab\Desktop\Projet_R\gadm41_SEN_1.shp' using driver `ESRI Shapefile'
## Simple feature collection with 14 features and 11 fields
## Geometry type: MULTIPOLYGON
## Dimension:     XY
## Bounding box:  xmin: -17.54319 ymin: 12.30786 xmax: -11.34247 ymax: 16.69207
## Geodetic CRS:  WGS 84
## Reading layer `gadm41_SEN_1' from data source 
##   `C:\Users\starlab\Desktop\Projet_R\gadm41_SEN_1.shp' using driver `ESRI Shapefile'
## Simple feature collection with 14 features and 11 fields
## Geometry type: MULTIPOLYGON
## Dimension:     XY
## Bounding box:  xmin: -17.54319 ymin: 12.30786 xmax: -11.34247 ymax: 16.69207
## Geodetic CRS:  WGS 84
\end{verbatim}

\includegraphics{Rapport_projet_files/figure-latex/unnamed-chunk-23-1.pdf}

\hypertarget{repruxe9sentation-spatiale-des-pme-suivant-le-niveau-dinstruction.}{%
\subsubsection{Représentation spatiale des PME suivant le niveau
d'instruction.}\label{repruxe9sentation-spatiale-des-pme-suivant-le-niveau-dinstruction.}}

\begin{Shaded}
\begin{Highlighting}[]
\FunctionTok{ggplot}\NormalTok{() }\SpecialCharTok{+}
  \FunctionTok{geom\_sf}\NormalTok{(}\AttributeTok{data=}\FunctionTok{st\_read}\NormalTok{(}\StringTok{"gadm41\_SEN\_1.shp"}\NormalTok{))}\SpecialCharTok{+}
  \FunctionTok{geom\_sf\_text}\NormalTok{(}\AttributeTok{data=}\FunctionTok{st\_read}\NormalTok{(}\StringTok{"gadm41\_SEN\_1.shp"}\NormalTok{), }\FunctionTok{aes}\NormalTok{(}\AttributeTok{label=}\NormalTok{NAME\_1))}\SpecialCharTok{+}
  \FunctionTok{geom\_sf}\NormalTok{(}\AttributeTok{data=}\NormalTok{projet\_map, }\FunctionTok{aes}\NormalTok{(}\AttributeTok{color=}\NormalTok{q25), }\AttributeTok{size=}\DecValTok{2}\NormalTok{)}\SpecialCharTok{+}
  \FunctionTok{labs}\NormalTok{(}\AttributeTok{title =} \StringTok{"Repartition des PME suivant le niveau d\textquotesingle{}instructions"}\NormalTok{,}
       \AttributeTok{subtitle =} \StringTok{"Carte du Sénégal"}\NormalTok{,}
       \AttributeTok{color =} \StringTok{"Niveau d\textquotesingle{}instruction"}\NormalTok{,}\AttributeTok{x =} \ConstantTok{NULL}\NormalTok{, }\AttributeTok{y =} \ConstantTok{NULL}\NormalTok{) }\SpecialCharTok{+}
  \FunctionTok{theme\_minimal}\NormalTok{() }\SpecialCharTok{+}
  \FunctionTok{theme}\NormalTok{(}
    \AttributeTok{plot.title =} \FunctionTok{element\_text}\NormalTok{(}\AttributeTok{hjust =} \DecValTok{1}\NormalTok{),}
    \AttributeTok{plot.subtitle =} \FunctionTok{element\_text}\NormalTok{(}\AttributeTok{hjust =} \DecValTok{1}\NormalTok{))}\SpecialCharTok{+}
  \FunctionTok{theme\_void}\NormalTok{()}\SpecialCharTok{+}
  \FunctionTok{annotation\_north\_arrow}\NormalTok{(}\AttributeTok{location =} \StringTok{"tl"}\NormalTok{, }\AttributeTok{scale =} \FloatTok{0.05}\NormalTok{)}
\end{Highlighting}
\end{Shaded}

\begin{verbatim}
## Reading layer `gadm41_SEN_1' from data source 
##   `C:\Users\starlab\Desktop\Projet_R\gadm41_SEN_1.shp' using driver `ESRI Shapefile'
## Simple feature collection with 14 features and 11 fields
## Geometry type: MULTIPOLYGON
## Dimension:     XY
## Bounding box:  xmin: -17.54319 ymin: 12.30786 xmax: -11.34247 ymax: 16.69207
## Geodetic CRS:  WGS 84
## Reading layer `gadm41_SEN_1' from data source 
##   `C:\Users\starlab\Desktop\Projet_R\gadm41_SEN_1.shp' using driver `ESRI Shapefile'
## Simple feature collection with 14 features and 11 fields
## Geometry type: MULTIPOLYGON
## Dimension:     XY
## Bounding box:  xmin: -17.54319 ymin: 12.30786 xmax: -11.34247 ymax: 16.69207
## Geodetic CRS:  WGS 84
\end{verbatim}

\includegraphics{Rapport_projet_files/figure-latex/unnamed-chunk-24-1.pdf}

\hypertarget{analyse-spatiale-de-notre-choix-analyse-spatiale-de-luxe9tat-des-routes-passant-devant-les-entreprises-.}{%
\subsubsection{Analyse spatiale de notre choix : Analyse spatiale de
l'état des routes passant devant les entreprises
.}\label{analyse-spatiale-de-notre-choix-analyse-spatiale-de-luxe9tat-des-routes-passant-devant-les-entreprises-.}}

\begin{Shaded}
\begin{Highlighting}[]
\FunctionTok{ggplot}\NormalTok{() }\SpecialCharTok{+}
  \FunctionTok{geom\_sf}\NormalTok{(}\AttributeTok{data=}\FunctionTok{st\_read}\NormalTok{(}\StringTok{"gadm41\_SEN\_1.shp"}\NormalTok{))}\SpecialCharTok{+}
  \FunctionTok{geom\_sf\_text}\NormalTok{(}\AttributeTok{data=}\FunctionTok{st\_read}\NormalTok{(}\StringTok{"gadm41\_SEN\_1.shp"}\NormalTok{), }\FunctionTok{aes}\NormalTok{(}\AttributeTok{label=}\NormalTok{NAME\_1))}\SpecialCharTok{+}
  \FunctionTok{geom\_sf}\NormalTok{(}\AttributeTok{data=}\NormalTok{projet\_map, }\FunctionTok{aes}\NormalTok{(}\AttributeTok{color=}\NormalTok{q17), }\AttributeTok{size=}\DecValTok{2}\NormalTok{)}\SpecialCharTok{+}
  \FunctionTok{labs}\NormalTok{(}\AttributeTok{title =} \StringTok{"état de la route bitumée passant devant les PME "}\NormalTok{,}
       \AttributeTok{subtitle =} \StringTok{"Carte du Sénégal"}\NormalTok{,}
       \AttributeTok{color =} \StringTok{"état de la route bitumée"}\NormalTok{,}\AttributeTok{x =} \ConstantTok{NULL}\NormalTok{, }\AttributeTok{y =} \ConstantTok{NULL}\NormalTok{) }\SpecialCharTok{+}
  \FunctionTok{theme\_minimal}\NormalTok{() }\SpecialCharTok{+}
  \FunctionTok{theme}\NormalTok{(}
    \AttributeTok{plot.title =} \FunctionTok{element\_text}\NormalTok{(}\AttributeTok{hjust =} \DecValTok{1}\NormalTok{),}
    \AttributeTok{plot.subtitle =} \FunctionTok{element\_text}\NormalTok{(}\AttributeTok{hjust =} \DecValTok{1}\NormalTok{))}\SpecialCharTok{+}
  \FunctionTok{theme\_void}\NormalTok{()}\SpecialCharTok{+}
  \FunctionTok{annotation\_north\_arrow}\NormalTok{(}\AttributeTok{location =} \StringTok{"tl"}\NormalTok{, }\AttributeTok{scale =} \DecValTok{5}\NormalTok{)}
\end{Highlighting}
\end{Shaded}

\begin{verbatim}
## Reading layer `gadm41_SEN_1' from data source 
##   `C:\Users\starlab\Desktop\Projet_R\gadm41_SEN_1.shp' using driver `ESRI Shapefile'
## Simple feature collection with 14 features and 11 fields
## Geometry type: MULTIPOLYGON
## Dimension:     XY
## Bounding box:  xmin: -17.54319 ymin: 12.30786 xmax: -11.34247 ymax: 16.69207
## Geodetic CRS:  WGS 84
## Reading layer `gadm41_SEN_1' from data source 
##   `C:\Users\starlab\Desktop\Projet_R\gadm41_SEN_1.shp' using driver `ESRI Shapefile'
## Simple feature collection with 14 features and 11 fields
## Geometry type: MULTIPOLYGON
## Dimension:     XY
## Bounding box:  xmin: -17.54319 ymin: 12.30786 xmax: -11.34247 ymax: 16.69207
## Geodetic CRS:  WGS 84
\end{verbatim}

\includegraphics{Rapport_projet_files/figure-latex/unnamed-chunk-25-1.pdf}

\hypertarget{section-1}{%
\section{\texorpdfstring{\textcolor{blue}{Partie 2}}{}}\label{section-1}}

\hypertarget{i.-nettoyage-et-gestion-des-donnuxe9es.}{%
\subsection{I. Nettoyage et gestion des
données.}\label{i.-nettoyage-et-gestion-des-donnuxe9es.}}

\begin{Shaded}
\begin{Highlighting}[]
\CommentTok{\#Importation de la base Base\_Partie 2.xlsx }
\NormalTok{partie2}\OtherTok{\textless{}{-}}\NormalTok{readxl}\SpecialCharTok{::}\FunctionTok{read\_xlsx}\NormalTok{(}\StringTok{"Base\_Partie 2.xlsx"}\NormalTok{)}
\end{Highlighting}
\end{Shaded}

\hypertarget{renommons-la-variable-country_destination-en-destination-et-duxe9finissons-les-valeurs-nuxe9gatives-comme-manquantes.}{%
\subsubsection{Renommons la variable ``country\_destination'' en
``destination'' et définissons les valeurs négatives comme
manquantes.}\label{renommons-la-variable-country_destination-en-destination-et-duxe9finissons-les-valeurs-nuxe9gatives-comme-manquantes.}}

\begin{Shaded}
\begin{Highlighting}[]
\CommentTok{\# Renommons la variable "country\_destination" en "destination"}
\NormalTok{partie2 }\OtherTok{\textless{}{-}} \FunctionTok{rename}\NormalTok{(partie2, }\AttributeTok{destination =}\NormalTok{ country\_destination)}
\CommentTok{\#Remplacement par des valeurs manquantes}
\NormalTok{partie2}\SpecialCharTok{$}\NormalTok{destination }\OtherTok{\textless{}{-}} \FunctionTok{ifelse}\NormalTok{(partie2}\SpecialCharTok{$}\NormalTok{destination }\SpecialCharTok{\textless{}} \DecValTok{0}\NormalTok{, }\ConstantTok{NA}\NormalTok{, partie2}\SpecialCharTok{$}\NormalTok{destination)}
\end{Highlighting}
\end{Shaded}

\hypertarget{cruxe9ation-dune-nouvelle-variable-contenant-les-tranches-duxe2ge-de-5-ans-en-utilisant-la-variable-age.}{%
\subsubsection{Création d'une nouvelle variable contenant les tranches
d'âge de 5 ans en utilisant la variable
``age''.}\label{cruxe9ation-dune-nouvelle-variable-contenant-les-tranches-duxe2ge-de-5-ans-en-utilisant-la-variable-age.}}

La base de donnée contenant des valeurs abbérantes pour la variable age
(999), nous imputons à toutes les valeurs supérieures à 100 la médiane
des ages.

\begin{Shaded}
\begin{Highlighting}[]
\CommentTok{\#Imputation aux valeurs abbérantes de la variable age de la médiane de la série}
\NormalTok{median\_age }\OtherTok{\textless{}{-}} \FunctionTok{median}\NormalTok{(}\FunctionTok{subset}\NormalTok{(partie2, age}\SpecialCharTok{\textless{}}\DecValTok{100}\NormalTok{)}\SpecialCharTok{$}\NormalTok{age)}
\NormalTok{partie2 }\OtherTok{\textless{}{-}}\NormalTok{ partie2}\SpecialCharTok{\%\textgreater{}\%}
  \FunctionTok{mutate}\NormalTok{(}\AttributeTok{age =} \FunctionTok{ifelse}\NormalTok{(age }\SpecialCharTok{\textless{}} \DecValTok{0} \SpecialCharTok{|}\NormalTok{ age }\SpecialCharTok{\textgreater{}} \DecValTok{100}\NormalTok{, }\FunctionTok{median}\NormalTok{(}\FunctionTok{subset}\NormalTok{(partie2, age}\SpecialCharTok{\textgreater{}}\DecValTok{100}\NormalTok{)}\SpecialCharTok{$}\NormalTok{age), age))}
\CommentTok{\#suppressions de l\textquotesingle{}objet après utilisation}
\FunctionTok{rm}\NormalTok{(median\_age)}
\end{Highlighting}
\end{Shaded}

Création de la variable proprement dite

\begin{Shaded}
\begin{Highlighting}[]
\CommentTok{\# Création de la nouvelle variable "age\_group" en découpant la variable "age" en tranches de 5 ans}
\NormalTok{age\_intervals }\OtherTok{\textless{}{-}} \FunctionTok{seq}\NormalTok{(}\DecValTok{0}\NormalTok{, }\DecValTok{100}\NormalTok{, }\AttributeTok{by =} \DecValTok{5}\NormalTok{)}
\NormalTok{age\_labels }\OtherTok{\textless{}{-}} \FunctionTok{paste0}\NormalTok{(}\StringTok{"]"}\NormalTok{, age\_intervals[}\SpecialCharTok{{-}}\FunctionTok{length}\NormalTok{(age\_intervals)], }\StringTok{", "}\NormalTok{, age\_intervals[}\SpecialCharTok{{-}}\DecValTok{1}\NormalTok{], }\StringTok{"]"}\NormalTok{)}

\NormalTok{partie2 }\OtherTok{\textless{}{-}}\NormalTok{ partie2 }\SpecialCharTok{\%\textgreater{}\%}
  \FunctionTok{mutate}\NormalTok{(}\AttributeTok{age\_interval =} \FunctionTok{cut}\NormalTok{(age, }\AttributeTok{breaks =}\NormalTok{ age\_intervals, }\AttributeTok{labels =}\NormalTok{ age\_labels, }\AttributeTok{right =} \ConstantTok{FALSE}\NormalTok{))}
\CommentTok{\#supression des objets après utilisation}
\FunctionTok{rm}\NormalTok{(age\_intervals, age\_intervals)}
\end{Highlighting}
\end{Shaded}

\hypertarget{cruxe9ation-dune-nouvelle-variable-contenant-le-nombre-dentretiens-ruxe9alisuxe9s-par-chaque-agent-recenseur}{%
\subsubsection{Création d'une nouvelle variable contenant le nombre
d'entretiens réalisés par chaque agent
recenseur}\label{cruxe9ation-dune-nouvelle-variable-contenant-le-nombre-dentretiens-ruxe9alisuxe9s-par-chaque-agent-recenseur}}

\begin{Shaded}
\begin{Highlighting}[]
\CommentTok{\# Variable nombre d\textquotesingle{}entretien par agent recenceur}
\NormalTok{partie2}\OtherTok{\textless{}{-}}\NormalTok{ partie2 }\SpecialCharTok{\%\textgreater{}\%} \FunctionTok{group\_by}\NormalTok{(enumerator) }\SpecialCharTok{\%\textgreater{}\%} \FunctionTok{mutate}\NormalTok{(}\AttributeTok{nbr\_entretien =} \FunctionTok{n}\NormalTok{())}
\CommentTok{\#  retirons les regroupements temporaires effectués par group\_by()}
\NormalTok{partie2 }\OtherTok{\textless{}{-}} \FunctionTok{ungroup}\NormalTok{(partie2)}
\end{Highlighting}
\end{Shaded}

\hypertarget{cruxe9ation-dune-nouvelle-variable-qui-affecte-aluxe9atoirement-chaque-ruxe9pondant-uxe0-un-groupe-de-traitement-1-ou-de-controle-0.}{%
\subsubsection{Création d'une nouvelle variable qui affecte
aléatoirement chaque répondant à un groupe de traitement (1) ou de
controle
(0).}\label{cruxe9ation-dune-nouvelle-variable-qui-affecte-aluxe9atoirement-chaque-ruxe9pondant-uxe0-un-groupe-de-traitement-1-ou-de-controle-0.}}

\begin{Shaded}
\begin{Highlighting}[]
\CommentTok{\#création de la variable aléatoire }
\NormalTok{partie2}\OtherTok{\textless{}{-}}\NormalTok{ partie2 }\SpecialCharTok{\%\textgreater{}\%} \FunctionTok{mutate}\NormalTok{(}\AttributeTok{group\_traitement =}\FunctionTok{sample}\NormalTok{(}\DecValTok{0}\SpecialCharTok{:}\DecValTok{1}\NormalTok{, }\FunctionTok{nrow}\NormalTok{(partie2), }
                                                      \AttributeTok{replace =} \ConstantTok{TRUE}\NormalTok{) )}
\end{Highlighting}
\end{Shaded}

\hypertarget{fusion-de-la-taille-de-la-population-de-chaque-district-feuille-2-avec-lensemble-de-donnuxe9es-feuille-1-afin-que-toutes-les-personnes-interroguxe9es-aient-une-valeur-correspondante-repruxe9sentant-la-taille-de-la-population-du-district-dans-lequel-elles-vivent.}{%
\subsubsection{Fusion de la taille de la population de chaque district
(feuille 2) avec l'ensemble de données (feuille 1) afin que toutes les
personnes interrogées aient une valeur correspondante représentant la
taille de la population du district dans lequel elles
vivent.}\label{fusion-de-la-taille-de-la-population-de-chaque-district-feuille-2-avec-lensemble-de-donnuxe9es-feuille-1-afin-que-toutes-les-personnes-interroguxe9es-aient-une-valeur-correspondante-repruxe9sentant-la-taille-de-la-population-du-district-dans-lequel-elles-vivent.}}

\begin{Shaded}
\begin{Highlighting}[]
\CommentTok{\# fusion avec la taille de la population}
\CommentTok{\# chargement la feuille disrict }
\NormalTok{district }\OtherTok{\textless{}{-}} \FunctionTok{read\_excel}\NormalTok{(}\StringTok{"Base\_Partie 2.xlsx"}\NormalTok{,}\AttributeTok{sheet =} \StringTok{"district"}\NormalTok{)}
\NormalTok{fusion}\OtherTok{\textless{}{-}}\FunctionTok{merge}\NormalTok{(partie2, district , }\AttributeTok{by=}\StringTok{"district"}\NormalTok{)}
\end{Highlighting}
\end{Shaded}

\hypertarget{calcul-de-la-duruxe9e-de-lentretien-et-indiquer-la-duruxe9e-moyenne-de-lentretien-par-enquuxeateur.}{%
\subsubsection{Calcul de la durée de l'entretien et indiquer la durée
moyenne de l'entretien par
enquêteur.}\label{calcul-de-la-duruxe9e-de-lentretien-et-indiquer-la-duruxe9e-moyenne-de-lentretien-par-enquuxeateur.}}

\begin{Shaded}
\begin{Highlighting}[]
\CommentTok{\# Convertion des colonnes en format de date et heure}
\NormalTok{fusion}\SpecialCharTok{$}\NormalTok{starttime }\OtherTok{\textless{}{-}} \FunctionTok{as.POSIXct}\NormalTok{(fusion}\SpecialCharTok{$}\NormalTok{starttime)}
\NormalTok{fusion}\SpecialCharTok{$}\NormalTok{endtime }\OtherTok{\textless{}{-}} \FunctionTok{as.POSIXct}\NormalTok{(fusion}\SpecialCharTok{$}\NormalTok{endtime)}
\CommentTok{\# Calcul de la durée de l\textquotesingle{}entretien en minutes}
\NormalTok{fusion}\SpecialCharTok{$}\NormalTok{duree\_entretien }\OtherTok{\textless{}{-}} \FunctionTok{as.numeric}\NormalTok{(}\FunctionTok{difftime}\NormalTok{(fusion}\SpecialCharTok{$}\NormalTok{endtime, fusion}\SpecialCharTok{$}\NormalTok{starttime, }\AttributeTok{units =} \StringTok{"mins"}\NormalTok{))}

\CommentTok{\# Calcul de la durée moyenne de l\textquotesingle{}entretien par enquêteur}
\NormalTok{duree\_moy\_p\_enqueteur }\OtherTok{\textless{}{-}} \FunctionTok{tapply}\NormalTok{(fusion}\SpecialCharTok{$}\NormalTok{duree\_entretien, fusion}\SpecialCharTok{$}\NormalTok{enumerator, mean)}
\end{Highlighting}
\end{Shaded}

\hypertarget{renommez-toutes-les-variables-de-lensemble-de-donnuxe9es-en-ajoutant-le-pruxe9fixe-endline_-uxe0-laide-dune-boucle.}{%
\subsubsection{Renommez toutes les variables de l'ensemble de données en
ajoutant le préfixe ``endline\_'' à l'aide d'une
boucle.}\label{renommez-toutes-les-variables-de-lensemble-de-donnuxe9es-en-ajoutant-le-pruxe9fixe-endline_-uxe0-laide-dune-boucle.}}

\begin{Shaded}
\begin{Highlighting}[]
\CommentTok{\# Renommons les variables avec le préfixe "endline\_"}
\NormalTok{newb}\OtherTok{\textless{}{-}}\NormalTok{fusion}
\CommentTok{\#Définition de la fonction de renommage et application grâce à un apply}
\NormalTok{newb }\OtherTok{\textless{}{-}} \FunctionTok{lapply}\NormalTok{(}\FunctionTok{names}\NormalTok{(newb), }\ControlFlowTok{function}\NormalTok{(var) \{}
  \ControlFlowTok{if}\NormalTok{ (}\SpecialCharTok{!}\FunctionTok{startsWith}\NormalTok{(var, }\StringTok{"endline\_"}\NormalTok{)) \{}
    \FunctionTok{names}\NormalTok{(newb)[}\FunctionTok{names}\NormalTok{(newb) }\SpecialCharTok{==}\NormalTok{ var] }\OtherTok{\textless{}{-}} \FunctionTok{paste0}\NormalTok{(}\StringTok{"endline\_"}\NormalTok{, var)}
\NormalTok{  \}}
  \FunctionTok{return}\NormalTok{(newb[[var]])}
\NormalTok{\})}
\CommentTok{\# Convertion la liste en un nouveau data frame avec les variables renommées}
\NormalTok{newb }\OtherTok{\textless{}{-}} \FunctionTok{as.data.frame}\NormalTok{(newb)}
\end{Highlighting}
\end{Shaded}

\hypertarget{ii.-analyse-et-visualisation-des-donnuxe9es.}{%
\subsection{II. Analyse et visualisation des
données.}\label{ii.-analyse-et-visualisation-des-donnuxe9es.}}

\hypertarget{cruxe9ation-dun-tableau-ruxe9capitulatif-contenant-luxe2ge-moyen-et-le-nombre-moyen-denfants-par-district.}{%
\subsubsection{Création d'un tableau récapitulatif contenant l'âge moyen
et le nombre moyen d'enfants par
district.}\label{cruxe9ation-dun-tableau-ruxe9capitulatif-contenant-luxe2ge-moyen-et-le-nombre-moyen-denfants-par-district.}}

\begin{Shaded}
\begin{Highlighting}[]
\CommentTok{\# Calcul du 1er et le 3e quartile de la variable ma\_variable}
\NormalTok{Q1 }\OtherTok{\textless{}{-}} \FunctionTok{quantile}\NormalTok{(partie2}\SpecialCharTok{$}\NormalTok{age, }\FloatTok{0.25}\NormalTok{)}
\NormalTok{Q3 }\OtherTok{\textless{}{-}} \FunctionTok{quantile}\NormalTok{(partie2}\SpecialCharTok{$}\NormalTok{age, }\FloatTok{0.75}\NormalTok{)}

\CommentTok{\# Calcul de l\textquotesingle{}écart interquartile (IQR)}
\NormalTok{IQR }\OtherTok{\textless{}{-}}\NormalTok{ Q3 }\SpecialCharTok{{-}}\NormalTok{ Q1}

\CommentTok{\# Définition  des seuils pour exclure les valeurs aberrantes (par exemple, 1,5 fois l\textquotesingle{}IQR)}
\NormalTok{seuil\_superieur }\OtherTok{\textless{}{-}}\NormalTok{ Q3 }\SpecialCharTok{+} \FloatTok{1.5} \SpecialCharTok{*}\NormalTok{ IQR}
\NormalTok{seuil\_inferieur }\OtherTok{\textless{}{-}}\NormalTok{ Q1 }\SpecialCharTok{{-}} \FloatTok{1.5} \SpecialCharTok{*}\NormalTok{ IQR}

\CommentTok{\# Filtrer les données pour exclure les valeurs aberrantes}
\NormalTok{donne\_filtre }\OtherTok{\textless{}{-}}\NormalTok{ partie2 }\SpecialCharTok{\%\textgreater{}\%} \FunctionTok{filter}\NormalTok{(age }\SpecialCharTok{\textgreater{}=}\NormalTok{ seuil\_inferieur, age }\SpecialCharTok{\textless{}=}\NormalTok{ seuil\_superieur)}

\NormalTok{tableau\_1 }\OtherTok{\textless{}{-}}\NormalTok{ donne\_filtre }\SpecialCharTok{\%\textgreater{}\%}\NormalTok{dplyr}\SpecialCharTok{::}\FunctionTok{select}\NormalTok{(district,age,children\_num)}\SpecialCharTok{\%\textgreater{}\%}
\NormalTok{  gtsummary}\SpecialCharTok{::}\FunctionTok{tbl\_continuous}\NormalTok{(}\AttributeTok{variable =}\NormalTok{age,}
                            \AttributeTok{statistic =}\SpecialCharTok{\textasciitilde{}} \StringTok{"\{mean\}"}\NormalTok{,}
                            \CommentTok{\#digits = 0,}
                            \AttributeTok{include=}\NormalTok{district)}
\NormalTok{tableau\_2 }\OtherTok{\textless{}{-}}\NormalTok{ donne\_filtre }\SpecialCharTok{\%\textgreater{}\%}\NormalTok{dplyr}\SpecialCharTok{::}\FunctionTok{select}\NormalTok{(district,age,children\_num)}\SpecialCharTok{\%\textgreater{}\%}
\NormalTok{  gtsummary}\SpecialCharTok{::}\FunctionTok{tbl\_continuous}\NormalTok{(}\AttributeTok{variable =}\NormalTok{children\_num,}
                            \AttributeTok{statistic =}\SpecialCharTok{\textasciitilde{}} \StringTok{"\{mean\}"}\NormalTok{,}
                            \CommentTok{\#digits = 0,}
                            \AttributeTok{include=}\NormalTok{district)}
\NormalTok{tableau\_recapitulatif}\OtherTok{\textless{}{-}}\FunctionTok{tbl\_merge}\NormalTok{(}\FunctionTok{list}\NormalTok{(tableau\_1,tableau\_2),}
                                 \AttributeTok{tab\_spanner=}\FunctionTok{c}\NormalTok{(}\StringTok{"Age Moyen des enfants"}\NormalTok{,}\StringTok{"Nombre Moyen d\textquotesingle{}enfant"}\NormalTok{))}\SpecialCharTok{\%\textgreater{}\%}
  \FunctionTok{modify\_caption}\NormalTok{(}\StringTok{"Age moyen \& nombre moyen d’enfants par district."}\NormalTok{)}\SpecialCharTok{\%\textgreater{}\%} \FunctionTok{bold\_labels}\NormalTok{()}
\NormalTok{tableau\_recapitulatif}
\end{Highlighting}
\end{Shaded}

\begin{longtable}[]{@{}lcc@{}}
\caption{Age moyen \& nombre moyen d'enfants par
district.}\tabularnewline
\toprule\noalign{}
\textbf{Characteristic} & \textbf{N = 94} & \textbf{N = 94} \\
\midrule\noalign{}
\endfirsthead
\toprule\noalign{}
\textbf{Characteristic} & \textbf{N = 94} & \textbf{N = 94} \\
\midrule\noalign{}
\endhead
\bottomrule\noalign{}
\endlastfoot
\textbf{district} & & \\
1 & 27.7 & 1.00 \\
2 & 26.6 & 0.69 \\
3 & 26.1 & 0.00 \\
4 & 26.0 & 0.00 \\
5 & 24.3 & 0.50 \\
6 & 23.2 & 0.12 \\
7 & 25.2 & 0.00 \\
8 & 24.6 & 1.27 \\
\end{longtable}

\hypertarget{test-de-si-la-diffuxe9rence-duxe2ge-entre-les-sexes-est-statistiquement-significative-au-niveau-de-5-.}{%
\subsubsection{Test de si la différence d'âge entre les sexes est
statistiquement significative au niveau de 5
\%.}\label{test-de-si-la-diffuxe9rence-duxe2ge-entre-les-sexes-est-statistiquement-significative-au-niveau-de-5-.}}

\begin{Shaded}
\begin{Highlighting}[]
\NormalTok{data\_test}\OtherTok{\textless{}{-}}\NormalTok{partie2}\SpecialCharTok{\%\textgreater{}\%} \FunctionTok{select}\NormalTok{(sex, intention)}
\CommentTok{\# Effectuons l\textquotesingle{}analyse de régression linéaire}
\NormalTok{modele }\OtherTok{\textless{}{-}} \FunctionTok{lm}\NormalTok{(intention }\SpecialCharTok{\textasciitilde{}}\NormalTok{ sex, }\AttributeTok{data =}\NormalTok{ data\_test)}

\CommentTok{\# Afficher les résultats du modèle}
\NormalTok{modele}\SpecialCharTok{\%\textgreater{}\%}\NormalTok{gtsummary}\SpecialCharTok{::}\FunctionTok{tbl\_regression}\NormalTok{()}
\end{Highlighting}
\end{Shaded}

\begin{longtable}[]{@{}lccc@{}}
\toprule\noalign{}
\textbf{Characteristic} & \textbf{Beta} & \textbf{95\% CI} &
\textbf{p-value} \\
\midrule\noalign{}
\endhead
\bottomrule\noalign{}
\endlastfoot
sex & -0.92 & -2.0, 0.16 & 0.093 \\
\end{longtable}

\hypertarget{cruxe9ation-dun-nuage-de-points-de-luxe2ge-en-fonction-du-nombre-denfants.}{%
\subsubsection{Création d'un nuage de points de l'âge en fonction du
nombre
d'enfants.}\label{cruxe9ation-dun-nuage-de-points-de-luxe2ge-en-fonction-du-nombre-denfants.}}

\begin{Shaded}
\begin{Highlighting}[]
\CommentTok{\# Création du nuage de points avec des effets visuels}
\NormalTok{nuage\_points }\OtherTok{\textless{}{-}} \FunctionTok{ggplot}\NormalTok{(}\AttributeTok{data =}\NormalTok{ donne\_filtre, }\FunctionTok{aes}\NormalTok{(}\AttributeTok{x =}\NormalTok{ children\_num, }\AttributeTok{y =}\NormalTok{ age)) }\SpecialCharTok{+}
  \FunctionTok{geom\_point}\NormalTok{(}\AttributeTok{color =} \StringTok{"black"}\NormalTok{, }\AttributeTok{size =} \DecValTok{3}\NormalTok{, }\AttributeTok{alpha =} \FloatTok{0.6}\NormalTok{) }\SpecialCharTok{+}  \CommentTok{\# Couleur, taille et transparence des points}
  \FunctionTok{labs}\NormalTok{(}\AttributeTok{title =} \StringTok{"Nuage de points de l\textquotesingle{}âge en fonction du nombre d\textquotesingle{}enfants"}\NormalTok{,}
       \AttributeTok{x =} \StringTok{"Nombre d\textquotesingle{}enfants"}\NormalTok{,}
       \AttributeTok{y =} \StringTok{"Âge"}\NormalTok{) }\SpecialCharTok{+}
  \FunctionTok{theme\_minimal}\NormalTok{() }\SpecialCharTok{+}  \CommentTok{\# Utiliser un thème minimal}
  \FunctionTok{theme}\NormalTok{(}\AttributeTok{panel.grid.major =} \FunctionTok{element\_line}\NormalTok{(}\AttributeTok{color =} \StringTok{"\#FFF0FF"}\NormalTok{, }\AttributeTok{linetype =} \StringTok{"dashed"}\NormalTok{),  }\CommentTok{\# Ajouter une grille en pointillés}
        \AttributeTok{panel.grid.minor =} \FunctionTok{element\_blank}\NormalTok{(),  }\CommentTok{\# Masquer les lignes de la grille secondaire}
        \AttributeTok{axis.text =} \FunctionTok{element\_text}\NormalTok{(}\AttributeTok{size =} \DecValTok{12}\NormalTok{),  }\CommentTok{\# Taille du texte des étiquettes d\textquotesingle{}axe}
        \AttributeTok{axis.title =} \FunctionTok{element\_text}\NormalTok{(}\AttributeTok{size =} \DecValTok{14}\NormalTok{, }\AttributeTok{face =} \StringTok{"bold"}\NormalTok{),  }\CommentTok{\# Taille et style du texte des titres d\textquotesingle{}axe}
        \AttributeTok{plot.title =} \FunctionTok{element\_text}\NormalTok{(}\AttributeTok{size =} \DecValTok{16}\NormalTok{, }\AttributeTok{face =} \StringTok{"bold"}\NormalTok{),  }\CommentTok{\# Taille et style du titre du graphique}
        \AttributeTok{legend.position =} \StringTok{"bottom"}\NormalTok{)  }\CommentTok{\# Positionner la légende en bas du graphique}

\CommentTok{\# Afficher le nuage de points }
\FunctionTok{print}\NormalTok{(nuage\_points)}
\end{Highlighting}
\end{Shaded}

\includegraphics{Rapport_projet_files/figure-latex/unnamed-chunk-37-1.pdf}

\hypertarget{estimation-de-leffet-de-lappartenance-au-groupe-de-traitement-sur-lintention-de-migrer.}{%
\subsubsection{Estimation de l'effet de l'appartenance au groupe de
traitement sur l'intention de
migrer.}\label{estimation-de-leffet-de-lappartenance-au-groupe-de-traitement-sur-lintention-de-migrer.}}

\begin{Shaded}
\begin{Highlighting}[]
\NormalTok{data\_test}\OtherTok{\textless{}{-}}\NormalTok{partie2}\SpecialCharTok{\%\textgreater{}\%}\NormalTok{ dplyr}\SpecialCharTok{::}\FunctionTok{select}\NormalTok{(intention,group\_traitement)}
\CommentTok{\# Effectuer l\textquotesingle{}analyse de régression linéaire}
\NormalTok{modele }\OtherTok{\textless{}{-}} \FunctionTok{lm}\NormalTok{(intention }\SpecialCharTok{\textasciitilde{}}\NormalTok{ group\_traitement, }\AttributeTok{data =}\NormalTok{ data\_test)}

\CommentTok{\# Afficher les résultats du modèle}
\NormalTok{modele}\SpecialCharTok{\%\textgreater{}\%}\NormalTok{gtsummary}\SpecialCharTok{::}\FunctionTok{tbl\_regression}\NormalTok{()}
\end{Highlighting}
\end{Shaded}

\begin{longtable}[]{@{}lccc@{}}
\toprule\noalign{}
\textbf{Characteristic} & \textbf{Beta} & \textbf{95\% CI} &
\textbf{p-value} \\
\midrule\noalign{}
\endhead
\bottomrule\noalign{}
\endlastfoot
group\_traitement & -0.02 & -0.72, 0.68 & \textgreater0.9 \\
\end{longtable}

\hypertarget{cruxe9ation-dun-tableau-de-ruxe9gression-avec-3-moduxe8les.-la-variable-de-ruxe9sultat-est-toujours-intention.-moduxe8le-a-moduxe8le-vide---effet-du-traitement-sur-les-intentions.-moduxe8le-b-effet-du-traitement-sur-les-intentions-en-tenant-compte-de-luxe2ge-et-du-sexe.-moduxe8le-c-identique-au-moduxe8le-b-mais-en-contruxf4lant-le-district.-les-ruxe9sultats-des-trois-moduxe8les-doivent-uxeatre-affichuxe9s-dans-un-seul-tableau.}{%
\subsubsection{Création d'un tableau de régression avec 3 modèles. La
variable de résultat est toujours ``intention''. Modèle A : Modèle vide
- Effet du traitement sur les intentions. Modèle B : Effet du traitement
sur les intentions en tenant compte de l'âge et du sexe. Modèle C :
Identique au modèle B mais en contrôlant le district. Les résultats des
trois modèles doivent être affichés dans un seul
tableau.}\label{cruxe9ation-dun-tableau-de-ruxe9gression-avec-3-moduxe8les.-la-variable-de-ruxe9sultat-est-toujours-intention.-moduxe8le-a-moduxe8le-vide---effet-du-traitement-sur-les-intentions.-moduxe8le-b-effet-du-traitement-sur-les-intentions-en-tenant-compte-de-luxe2ge-et-du-sexe.-moduxe8le-c-identique-au-moduxe8le-b-mais-en-contruxf4lant-le-district.-les-ruxe9sultats-des-trois-moduxe8les-doivent-uxeatre-affichuxe9s-dans-un-seul-tableau.}}

\begin{Shaded}
\begin{Highlighting}[]
\NormalTok{data\_test}\OtherTok{\textless{}{-}}\NormalTok{partie2}\SpecialCharTok{\%\textgreater{}\%} \FunctionTok{select}\NormalTok{(intention,group\_traitement,age,sex,district)}
\CommentTok{\# Création des modèles de régression}
\NormalTok{modele\_A }\OtherTok{\textless{}{-}} \FunctionTok{lm}\NormalTok{(intention }\SpecialCharTok{\textasciitilde{}}\NormalTok{ group\_traitement, }\AttributeTok{data =}\NormalTok{ data\_test)}
\NormalTok{modele\_B }\OtherTok{\textless{}{-}} \FunctionTok{lm}\NormalTok{(intention }\SpecialCharTok{\textasciitilde{}}\NormalTok{ group\_traitement }\SpecialCharTok{+}\NormalTok{ age }\SpecialCharTok{+}\NormalTok{ sex, }\AttributeTok{data =}\NormalTok{ data\_test)}
\NormalTok{modele\_C }\OtherTok{\textless{}{-}} \FunctionTok{lm}\NormalTok{(intention }\SpecialCharTok{\textasciitilde{}}\NormalTok{ group\_traitement }\SpecialCharTok{+}\NormalTok{ age }\SpecialCharTok{+}\NormalTok{ sex }\SpecialCharTok{+}\NormalTok{ district,}
               \AttributeTok{data =}\NormalTok{ data\_test)}

\CommentTok{\# Création du tableau de régression avec gtsummary}
\NormalTok{tA}\OtherTok{\textless{}{-}}\NormalTok{modele\_A}\SpecialCharTok{\%\textgreater{}\%}\FunctionTok{tbl\_regression}\NormalTok{(}\AttributeTok{exponentiate =} \ConstantTok{FALSE}\NormalTok{)}
\NormalTok{tB}\OtherTok{\textless{}{-}}\NormalTok{modele\_B}\SpecialCharTok{\%\textgreater{}\%}\FunctionTok{tbl\_regression}\NormalTok{(}\AttributeTok{exponentiate =} \ConstantTok{FALSE}\NormalTok{)}
\NormalTok{tC}\OtherTok{\textless{}{-}}\NormalTok{modele\_C}\SpecialCharTok{\%\textgreater{}\%}\FunctionTok{tbl\_regression}\NormalTok{(}\AttributeTok{exponentiate =} \ConstantTok{FALSE}\NormalTok{)}
\NormalTok{tableau\_regression }\OtherTok{\textless{}{-}} \FunctionTok{tbl\_stack}\NormalTok{(}\FunctionTok{list}\NormalTok{(tA,tB,tC),}\AttributeTok{group\_header=}\FunctionTok{c}\NormalTok{(}\StringTok{"Modèle A"}\NormalTok{,}\StringTok{"Modèle B"}\NormalTok{,}\StringTok{"Modèle C"}\NormalTok{))}
\NormalTok{tableau\_regression}
\end{Highlighting}
\end{Shaded}

\begin{longtable}[]{@{}llccc@{}}
\toprule\noalign{}
\textbf{Group} & \textbf{Characteristic} & \textbf{Beta} & \textbf{95\%
CI} & \textbf{p-value} \\
\midrule\noalign{}
\endhead
\bottomrule\noalign{}
\endlastfoot
Modèle A & group\_traitement & -0.02 & -0.72, 0.68 & \textgreater0.9 \\
Modèle B & group\_traitement & 0.09 & -0.62, 0.79 & 0.8 \\
& age & 0.00 & 0.00, 0.00 & 0.9 \\
& sex & -0.92 & -2.1, 0.23 & 0.12 \\
Modèle C & group\_traitement & 0.14 & -0.57, 0.85 & 0.7 \\
& age & 0.00 & 0.00, 0.00 & \textgreater0.9 \\
& sex & -0.88 & -2.0, 0.27 & 0.13 \\
& district & 0.09 & -0.06, 0.24 & 0.3 \\
\end{longtable}

\hypertarget{textcolorbluepartie-3-r-shiny}{%
\section{\textbackslash textcolor\{blue\}\{Partie 3 :
R-shiny\}}\label{textcolorbluepartie-3-r-shiny}}

Dans cette partie, il est question de faire une application r shiny qui
permet:

\begin{itemize}
\item
  de visualiser les evenements par pays (le nombre d'evenement par pays
  dans une carte)
\item
  de visualiser les evenements par pays, type, annee et la localisation.
\end{itemize}

\hypertarget{section-2}{%
\subsubsection{\texorpdfstring{\underline{Idée de conception et fonctionnement de l'application}}{}}\label{section-2}}

Pour résoudre ces deux questions à la fois, nous donnons la possibilité
à l'utilisateur de sélectionner directement les variables (pays,
évenements, années) qui l'intéressent dans la base de donnée fournie par
l'exercice, et ceci grâce à l'interface web fournie par R-shiny. Il en
va de soi qu'avec ces données que l'utilisateur entre dans l'appli, nous
lui retournons l'emplacement géographique de tous les évenements
sélectionnés, la carte du pays sélectionné et aussi l'année. Nous nous
servons de la légende qui est interactive pour afficher les différentes
statistiques concernant sa sélection.

Ainsi, lorsque l'utilisateur voudra avoir par exemple le nombre
d'évenements par pays, il lui suffira de sélectionner uniquement le pays
d'intérêt et de sélectionner tous les types d'évenements ainsi que
toutes les années.

\color{pink}{liste des packages utilisés dans la section R-shiny}

\begin{Shaded}
\begin{Highlighting}[]
\CommentTok{\#install.packages(c(leaflet, sf, shinydashboard, rnaturalearth, rnaturalearthdata, ))}
\end{Highlighting}
\end{Shaded}


\end{document}
